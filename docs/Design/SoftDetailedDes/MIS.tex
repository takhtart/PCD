\documentclass[12pt, titlepage]{article}

\usepackage{amsmath, mathtools}

\usepackage[round]{natbib}
\usepackage{amsfonts}
\usepackage{amssymb}
\usepackage{graphicx}
\usepackage{colortbl}
\usepackage{xr}
\usepackage{hyperref}
\usepackage{longtable}
\usepackage{xfrac}
\usepackage{tabularx}
\usepackage{float}
\usepackage{siunitx}
\usepackage{booktabs}
\usepackage{multirow}
\usepackage[section]{placeins}
\usepackage{caption}
\usepackage{fullpage}

\hypersetup{
bookmarks=true,     % show bookmarks bar?
colorlinks=true,       % false: boxed links; true: colored links
linkcolor=red,          % color of internal links (change box color with linkbordercolor)
citecolor=blue,      % color of links to bibliography
filecolor=magenta,  % color of file links
urlcolor=cyan          % color of external links
}

\usepackage{array}

\externaldocument{../../SRS/SRS}

%% Comments

\usepackage{color}

\newif\ifcomments\commentstrue %displays comments
%\newif\ifcomments\commentsfalse %so that comments do not display

\ifcomments
\newcommand{\authornote}[3]{\textcolor{#1}{[#3 ---#2]}}
\newcommand{\todo}[1]{\textcolor{red}{[TODO: #1]}}
\else
\newcommand{\authornote}[3]{}
\newcommand{\todo}[1]{}
\fi

\newcommand{\wss}[1]{\authornote{blue}{SS}{#1}} 
\newcommand{\plt}[1]{\authornote{magenta}{TPLT}{#1}} %For explanation of the template
\newcommand{\an}[1]{\authornote{cyan}{Author}{#1}}

%% Common Parts

\newcommand{\progname}{PCD: Partially Covered Detection of Obscured People using Point Cloud Data} % PUT YOUR PROGRAM NAME HERE
\newcommand{\authname}{Team \#14, PCD
\\ Tarnveer Takhtar
\\ Matthew Bradbury
\\ Harman Bassi
\\ Kyen So} % AUTHOR NAMES                  

\usepackage{hyperref}
    \hypersetup{colorlinks=true, linkcolor=blue, citecolor=blue, filecolor=blue,
                urlcolor=blue, unicode=false}
    \urlstyle{same}

\usepackage{indentfirst}                              


\begin{document}

\title{Module Interface Specification for \progname{}}

\author{\authname}

\date{January 17, 2025}

\maketitle

\pagenumbering{roman}

\section{Revision History}

\begin{tabularx}{\textwidth}{p{3cm}p{2cm}X}
\toprule {\bf Date} & {\bf Version} & {\bf Notes}\\
\midrule
Jan 17, 2025 & 1.0 & Initial Draft\\
\bottomrule
\end{tabularx}

~\newpage

\section{Symbols, Abbreviations and Acronyms}

SRS Documentation can be found on \href{https://github.com/takhtart/PCD/blob/main/docs/SRS/SRS.pdf}{GitHub}.\\
\\
\renewcommand{\arraystretch}{1.2}
\begin{tabular}{l l} 
  \toprule		
  \textbf{symbol} & \textbf{description}\\
  \midrule 
  SRS & Software Requirements Specification\\
  PCD & Partially Covered Detection Program \\
  PCL & Point Cloud Library \\
  \bottomrule
\end{tabular}\\

\newpage

\tableofcontents

\newpage

\pagenumbering{arabic}

\section{Introduction}

The following document details the Module Interface Specifications for Partially Covered Detection (PCD) software. 
The sections in this document describes each module in our software and how each module interacts with each other. 
Additional information and documentation can be found in System Requirement Specifications (SRS).

Complementary documents include the System Requirement Specifications
and Module Guide.  The full documentation and implementation can be
found at \url{https://github.com/takhtart/PCD}. 

\section{Notation}

The structure of the MIS for modules comes from \citet{HoffmanAndStrooper1995},
with the addition that template modules have been adapted from
\cite{GhezziEtAl2003}.  The mathematical notation comes from Chapter 3 of
\citet{HoffmanAndStrooper1995}.  For instance, the symbol := is used for a
multiple assignment statement and conditional rules follow the form $(c_1
\Rightarrow r_1 | c_2 \Rightarrow r_2 | ... | c_n \Rightarrow r_n )$.

The following table summarizes the primitive data types used by \progname. 

\begin{center}
\renewcommand{\arraystretch}{1.2}
\noindent 
\begin{tabular}{l l p{7.5cm}} 
\toprule 
\textbf{Data Type} & \textbf{Notation} & \textbf{Description}\\ 
\midrule
character & char & a single symbol or digit\\
integer & $\mathbb{Z}$ & a number without a fractional component in (-$\infty$, $\infty$) \\
natural number & $\mathbb{N}$ & a number without a fractional component in [1, $\infty$) \\
real & $\mathbb{R}$ & any number in (-$\infty$, $\infty$)\\
PointXYZRGBA & $\mathbb{P}$ & point cloud data in the PCL Library\\
\bottomrule
\end{tabular} 
\end{center}

\noindent
The specification of PCD uses some derived data types: sequences, strings, and
tuples. Sequences are lists filled with elements of the same data type. Strings
are sequences of characters. Tuples contain a list of values, potentially of
different types. In addition, PCD uses functions, which
are defined by the data types of their inputs and outputs. Local functions are
described by giving their type signature followed by their specification.

\section{Module Decomposition}

The following table is taken directly from the Module Guide document for this project.

\begin{table}[h!]
\centering
\begin{tabular}{p{0.3\textwidth} p{0.6\textwidth}}
\toprule
\textbf{Level 1} & \textbf{Level 2}\\
\midrule

{Hardware-Hiding Module} & Kinect Stream \\
\midrule

\multirow{7}{0.3\textwidth}{Behaviour-Hiding Module} & Application Control\\
& Input Data Read\\
& Input Classifier\\
& Input Classifier Ranking\\
& Bounding Box Display\\
\midrule

\multirow{3}{0.3\textwidth}{Software Decision Module} & Point Cloud Data Structures\\
& Input Processing\\
& Command Line Interface\\
& Graphical User Interface\\
\bottomrule

\end{tabular}
\caption{Module Hierarchy}
\label{TblMH}
\end{table}

\newpage
~\newpage

\section{MIS of Kinect Stream} \label{ModuleKS} 

\subsection{Module}

kinect

\subsection{Uses}

\begin{itemize}
\item Input Data Read \ref{ModuleIDR}
\item Point Cloud Data Structure \ref{ModulePCDS}
\end{itemize}

\subsection{Syntax}

\subsubsection{Exported Constants}

None.

\subsubsection{Exported Access Programs}

\begin{center}
\begin{tabular}{p{2cm} p{4cm} p{5cm} p{2cm}}
\hline
\textbf{Name} & \textbf{In} & \textbf{Out} & \textbf{Exceptions} \\
\hline
kinect & video frame stream from kinect in BGRA format and depth frame stream
& - pcl::PointXYZ & - \\
 & & - pcl::PointXYZRGBA &  \\
 & & - pcl::PointXYZI &  \\
 & & - pcl::PointXYZRGB &  \\
\hline
\end{tabular}
\end{center}

\subsection{Semantics}

\subsubsection{State Variables}

None

\subsubsection{Environment Variables}

Size of the room (affects \# of points in point cloud)

\subsubsection{Assumptions}

\begin{itemize}
  \item User selects live stream rather than offline view
  \item Kinect is connected and running without issue.
  \item Kinect has a clear and unobstructed view of the environment (lens are not covered)
\end{itemize}

\subsubsection{Access Routine Semantics}

\noindent kinect():
\begin{itemize}
  \item output: PointCloudT::ConstPtr\& input\_cloud
  \item Precondition: user calls live\_stream mode
  \item Postcondition: user terminates program or selects exit 
\end{itemize}

\subsubsection{Local Functions}

None

\newpage

\section{MIS of Application Control} \label{ModuleAC} 

\subsection{Module}

main

\subsection{Uses}

\begin{itemize}
  \item Input Data Read \ref{ModuleIDR}
  \item Input Classifier Module \ref{ModuleIC}
  \item Command Line Interface \ref{ModuleCLI}
  \item Graphical User Interface \ref{ModuleGUI}
  \item Bounding Box Display \ref{ModuleBBD}
\end{itemize}

\subsection{Syntax}

\subsubsection{Exported Constants}

None.

\subsubsection{Exported Access Programs}

\begin{center}
\begin{tabular}{p{2cm} p{4cm} p{4cm} p{2cm}}
\hline
\textbf{Name} & \textbf{In} & \textbf{Out} & \textbf{Exceptions} \\
\hline
main & None & None & - \\
\hline
\end{tabular}
\end{center}

\subsection{Semantics}

\subsubsection{State Variables}

None.

\subsubsection{Environment Variables}

\begin{itemize}
\item Processing speed of device
\end{itemize}

\subsubsection{Assumptions}

Device has the processing power needed.

\subsubsection{Access Routine Semantics}

\noindent main():
\begin{itemize}
\item transition:  connects the input data read module to input processing
\end{itemize}


\subsubsection{Local Functions}

None.

\newpage

\section{MIS of Input Data Read} \label{ModuleIDR} 

\subsection{Module}
reader
\subsection{Uses}

\subsection{Syntax}

\subsubsection{Exported Constants}

None.

\subsubsection{Exported Access Programs}

\begin{center}
\begin{tabular}{p{2cm} p{4cm} p{4cm} p{2cm}}
\hline
\textbf{Name} & \textbf{In} & \textbf{Out} & \textbf{Exceptions} \\
\hline
reader & - & - & - \\
\hline
\end{tabular}
\end{center}

\subsection{Semantics}

\subsubsection{State Variables}

\begin{itemize}
  \item Choice (Integer): Choice that the user made on what mode to run the program in.
\end{itemize}

\subsubsection{Environment Variables}

\begin{itemize}
  \item std::cin: Used to get the users input on what choice they want to make.
\end{itemize}

\subsubsection{Assumptions}

The user provides a valid input ($\mathbb{Z}$) corresponding with the correct option

\subsubsection{Access Routine Semantics}

\noindent main():
\begin{itemize}
  \item transition: Converts the input data into the data structure used by the Input Processing Module \ref{}
\end{itemize}

\subsubsection{Local Functions}

None.

\newpage

\section{MIS of Input Classifier} \label{ModuleIC} 

\subsection{Module}

classify

\subsection{Uses}


\begin{itemize}
  \item Bounding Box Display \ref{ModuleBBD}
  \item Point Cloud Data Structures \ref{ModulePCDS}
\end{itemize}

\subsection{Syntax}

\subsubsection{Exported Constants}

None.

\subsubsection{Exported Access Programs}

\begin{center}
\begin{tabular}{p{2cm} p{4cm} p{4cm} p{2cm}}
\hline
\textbf{Name} & \textbf{In} & \textbf{Out} & \textbf{Exceptions} \\
\hline
clssify & (PointCloudT::Ptr)cloud, (PointCloudT::Ptr) cloudfiltered & - & - \\
\hline
\end{tabular}
\end{center}

\subsection{Semantics}

\subsubsection{State Variables}

\begin{itemize}
  \item *cloudFiltered = *personCloud : updated cloud value for the data set
  \item personCloud (PointCloudT::Ptr) : the cliuster that is being identified as a human within the frame.
\end{itemize}

\subsubsection{Environment Variables}

None.

\subsubsection{Assumptions}

\begin{itemize}
  \item The cloud has been properly filtered to allow for a good reading to quickly identify the human on screen.
\end{itemize}

\subsubsection{Access Routine Semantics}

\noindent classify(cloud, cloudfiltered )():
transition: This module will take the the filtered values and filter down further to just identify the human within the frame and only leave the data points connected to that person.
\subsubsection{Local Functions}

None.
\newpage

\section{MIS of Input Classifier Ranking} \label{ModuleICR} 

\subsection{Module}

ranking

\subsection{Uses}

\subsection{Syntax}

\subsubsection{Exported Constants}

None.

\subsubsection{Exported Access Programs}

\begin{center}
\begin{tabular}{p{2cm} p{4cm} p{4cm} p{2cm}}
\hline
\textbf{Name} & \textbf{In} & \textbf{Out} & \textbf{Exceptions} \\
\hline
ranking & dataPoint($\mathbb{P}$)  & - & - \\
\hline
\end{tabular}
\end{center}

\subsection{Semantics}

\subsubsection{State Variables}

\begin{itemize}
  \item weights ($\mathbf{P}^{n}$): An array of size n containing the ordered weights of the pcd points
\end{itemize}

\subsubsection{Environment Variables}

None.

\subsubsection{Assumptions}

\begin{itemize}
  \item The input point cloud data is valid.
  \item The classification strategy is implemented correctly to be able to order the weights
\end{itemize}

\subsubsection{Access Routine Semantics}

\noindent ranking(dataPoint)():
\begin{itemize}
  \item transition: This module will take in the dataPoint and add it to the list and order it into the array.
\end{itemize}

\subsubsection{Local Functions}

None.
\newpage


\section{MIS of Bounding Box Display} \label{ModuleBBD} 

\subsection{Module}

boundingBox

\subsection{Uses}

\begin{itemize}
  \item Input Classifier Module
\end{itemize}

\subsection{Syntax}

\subsubsection{Exported Constants}

None.

\subsubsection{Exported Access Programs}

\begin{center}
\begin{tabular}{p{2cm} p{4cm} p{2cm} p{2cm}}
\hline
\textbf{Name} & \textbf{In} & \textbf{Out} & \textbf{Exceptions} \\
\hline
boundingBox & humancloud($\mathbb{P}$),
              minpt($\mathbb{P}$),
              maxpt($\mathbb{P}$) & - & - \\
\hline
\end{tabular}
\end{center}

\subsection{Semantics}

\subsubsection{State Variables}

\begin{itemize}
  \item thickness ($\mathbf{R}$): An float with the thickness of the box
  \item $scale_factor$ ($\mathbf{R}$): Factor that adjusts the box size.
\end{itemize}

\subsubsection{Environment Variables}

None.

\subsubsection{Assumptions}

\begin{itemize}
  \item The input cloud points are valid to provide an accurate drawing of the box.
\end{itemize}

\subsubsection{Access Routine Semantics}

\noindent boundingBox(humancloud,minpt,maxpt)():
transition: This module will take in inputed data points and add draw out a box using the max and pin points provided.
\subsubsection{Local Functions}

None.
\newpage

\section{MIS of Point Cloud Data Structures} \label{ModulePCDS} 

\subsection{Module}

struct

\subsection{Uses}

\begin{itemize}
  \item Input Processing \ref{ModuleIP}
  \item Graphical User Interface \ref{ModuleGUI}
\end{itemize}

\subsection{Syntax}

\subsubsection{Exported Constants}

None.

\subsubsection{Exported Access Programs}

\begin{center}
\begin{tabular}{p{2cm} p{4cm} p{4cm} p{2cm}}
\hline
\textbf{Name} & \textbf{In} & \textbf{Out} & \textbf{Exceptions} \\
\hline
struct & None & None & - \\
\hline
\end{tabular}
\end{center}

\subsection{Semantics}

\subsubsection{State Variables}

\begin{itemize}
\item typedef pcl::PointXYZRGBA PointT : Data structure designed to store each individual point in a point cloud 
\item typedef pcl::PointCloud\textless PointT\textgreater PointCloudT : Data Structure designed to store an entire point cloud
\end{itemize}

\subsubsection{Environment Variables}

None.

\subsubsection{Assumptions}

The point cloud data from kinect stream is valid

\subsubsection{Access Routine Semantics}

\noindent struct():
\begin{itemize}
\item transition: This module is the point cloud data structure for storing point cloud data such as ones captured from the kinect and ones processed by our cloud processing algorithm
\end{itemize}


\subsubsection{Local Functions}

None.

\newpage

\section{MIS of Input Processing} \label{ModuleIP} 


\subsection{Module}

processing

\subsection{Uses}

\begin{itemize}
\item Command Line Interface \ref{ModuleCLI}
\item Graphical User Interface \ref{ModuleGUI}
\end{itemize}

\subsection{Syntax}

\subsubsection{Exported Constants}

None.

\subsubsection{Exported Access Programs}

\begin{center}
\begin{tabular}{p{2cm} p{4cm} p{4cm} p{2cm}}
\hline
\textbf{Name} & \textbf{In} & \textbf{Out} & \textbf{Exceptions} \\
\hline
processing & PointCloudT::Ptr\& cloud & PointCloudT::Ptr\& cloud\_filtered & - \\
\hline
\end{tabular}
\end{center}

\subsection{Semantics}

\subsubsection{State Variables}

None.

\subsubsection{Environment Variables}

None.

\subsubsection{Assumptions}

Data captured by the kinect are correctly processed and stored in the point cloud data structure

\subsubsection{Access Routine Semantics}

\noindent process\_cloud(cloud,cloud\_filtered):
\begin{itemize}
\item output: performs voxel downsampling, remove noise, and perform euclidean cluster extraction of the input cloud and return a filtered version of the point cloud 
\end{itemize}


\subsubsection{Local Functions}

None.

\newpage

\section{MIS of Command Line Interface} \label{ModuleCLI} 

\subsection{Module}

cmd

\subsection{Uses}

\begin{itemize}
  \item Input Processing Module \ref{ModuleIP}
  \item Graphical User Interface \ref{ModuleGUI}
\end{itemize}



\subsection{Syntax}

\subsubsection{Exported Constants}

None.

\subsubsection{Exported Access Programs}

\begin{center}
\begin{tabular}{p{2cm} p{4cm} p{4cm} p{2cm}}
\hline
\textbf{Name} & \textbf{In} & \textbf{Out} & \textbf{Exceptions} \\
\hline
cmd & None. & None. & - \\
\hline
\end{tabular}
\end{center}

\subsection{Semantics}

\subsubsection{State Variables}

None.

\subsubsection{Environment Variables}

\begin{itemize}
  \item Keyboard (Used to select modes)
  \item Mouse (Interacts with command prompt)
\end{itemize}

\subsubsection{Assumptions}

None.

\subsubsection{Access Routine Semantics}

\noindent cmd():
\begin{itemize}
\item transition: Provides the interface that allows the user to select between live stream and offline mode 
\end{itemize}

\subsubsection{Local Functions}

None.

\newpage

\section{MIS of Graphical User Interface} \label{ModuleGUI} 

\subsection{Module}

gui

\subsection{Uses}

None.

\subsection{Syntax}

\subsubsection{Exported Constants}

None.

\subsubsection{Exported Access Programs}

\begin{center}
\begin{tabular}{p{2cm} p{4cm} p{4cm} p{2cm}}
\hline
\textbf{Name} & \textbf{In} & \textbf{Out} & \textbf{Exceptions} \\
\hline
gui & None. & None. & - \\
\hline
\end{tabular}
\end{center}

\subsection{Semantics}

\subsubsection{State Variables}

None.

\subsubsection{Environment Variables}

\begin{itemize}
  \item Mouse (To move around within the visualized point cloud)
\end{itemize}

\subsubsection{Assumptions}

Point cloud was correctly processed and stored with the point cloud data structure


\subsubsection{Access Routine Semantics}

\noindent gui():
\begin{itemize}
\item transition: Uses the visualizer to deploy a GUI which displays the 3D point cloud
\end{itemize}

\subsubsection{Local Functions}

None.

\newpage

\bibliographystyle {plainnat}
\bibliography {../../../refs/References}

\newpage

\section{Appendix} \label{Appendix}

N/A

\newpage{}

\section*{Appendix --- Reflection}

\begin{enumerate}
\item Why is it important to create a development plan prior to starting the
project?\\

It is important to create a development plan prior to starting the project as
it allows most of the heavy lifting behind project planning to be done before any
work has started. It allows for expectations and workflow to be clearly defined before
any issues arise. It also creates a document that can be referenced at other times 
to avoid confusion.

\item In your opinion, what are the advantages and disadvantages of using
CI/CD?\\

Employing CI/CD allows for better issue tracking and rollbacks. Utilizing CI/CD gives the
opportunity for teams to better track individual issues and commits, leading to increased 
awareness and visibility on workflow issues. It also allows for easier time rolling back
to a previous version in case something goes wrong. 
Some disadvantages are with the conceptual depth and speed. Ensuring a specific workflow 
and constant PR reviews can slow things down as contributors have to make sure that they
are following the workflow properly and have to wait for PR reviews (when necessary). Furthermore,
it is more effort to set up, both in the codebase and conceptually. The process has to be 
talked through and understood by all team members.

\item What disagreements did your group have in this deliverable, if any,
and how did you resolve them?\\

Our group mainly debated how to set up our GitHub workflows. Initially, we considered using individual branches for each issue, but we ultimately decided on a 
more streamlined approach with two revision branches and separate forks. This allows us to effectively manage pull requests for merging feature changes into the codebase. 
We reached this agreement after discussing the benefits of clarity and collaboration in our development process.

\item What went well while writing this deliverable? \\

This deliverable allowed for the group to be able to discuss standard goals vs stretched goals. Everyone contributed their ideas and we were able to come to a clear conclusion 
when it came to the goals and problem statement of the project. It also allowed us all to see where the project is headed and how we should properly prepare
 ourselves so that we can achieve our goals. 

\item What pain points did you experience during this deliverable, and how
did you resolve them?\\

The biggest pain point during this deliverable was being able to decide which goals were too ambitious and outside our design scope. Some goals such as the aspect of outlining the human 
in the environment were broken up. There was a deep discussion on how to properly decide the goals, but at the end the  group got a better understanding of the project.

\item How did you and your team adjust the scope of your goals to ensure
they are suitable for a Capstone project (not overly ambitious but also of
appropriate complexity for a senior design project)?\\

Because our project is presented by a professor, the team already had a pretty clear understanding of the goals that needed to be achieved for our project to be considered a success.
 The only issue was trying to ensure that the goals can be properly broken down so that a goal that seemed a bit ambitious could be broken into something that seems doable. For example, 
 the human detection is broken into the Minimal and viable product goal and then also extended into the stretch goal. This was an important discussion that the group had to ensure that we 
 deemed our goals to be doable.

\item What knowledge and skills will the team collectively need to acquire to successfully complete this capstone project?  Examples of possible knowledge to acquire include domain specific knowledge 
from the domain of your application, or software engineering knowledge, mechatronics knowledge or computer science knowledge.  Skills may be related to technology, or writing, or presentation, 
or team management, etc.  You should look to identify at least one item for each team member.

Every member of the group needs to acquire knowledge on Computer Vision and understanding how pcd files operate. Overall these are big topics and so the basics should be acquired by every member, but 
some specifics would be broken down to different parts for each member. Tarnveer and Matthew would be assigned to understanding how to track people on screen and map them on the screen. Harman and Kyen 
would be working on reading in the pcd files and understanding the PCL. The PCL will explore on aspects of boxing the points on screen. Tarnveer would also be responsible for understanding
the real time coding aspect in c++. Matthew would also be assigned to acquire knowledge on improving the human outline based on better data. Harman will be assigned to understanding how to cancel out noise
from the data set. Kyen will have to understand how to find the person based off the pcd and understand how to properly read the files.

\item For each of the knowledge areas and skills identified in the previous question, what are at least two approaches to acquiring the knowledge or mastering the skill?  Of the identified approaches, which
will each team member pursue, and why did they make this choice?

One approach for acquiring the knowledge would be to use the provided documentation for the libraries (PCL and OpenCV). This would provide a good fundamental understanding for the important aspects of the project. 

Another approach would be to just watch videos on the specific topics and try to understand from there. This would be able to provide a more visual explanations for the topics.

Tarnveer: use documentation and videos
Matthew: use documentation and videos 
Kyen: use documentation
Harman: use documentation and videos

\item What went well while writing this deliverable? 

The deliverable was straightforward. We had a rough idea of the main hazards within our project and 
tried to make sure that we covered the main scope. The document writing was split between all of us.
The document is pretty straightforward and we as a group were able to talk over the different sections
and divide up the work.

\item What pain points did you experience during this deliverable, and how
did you resolve them?

The biggest pain point was probably discussing what would be some assumptions we had to make, but we were able to come to an agreement by communicating our points of why or why not.

\item Which of your listed risks had your team thought of before this
deliverable, and which did you think of while doing this deliverable? For
the latter ones (ones you thought of while doing the Hazard Analysis), how
did they come about?

All the risks were mainly thought of before the deliverable. We knew that we needed to ensure that the 
offline file is the correct format and that the system needs to make sure it is working with a Kinect sensor and not something else.

The privacy risk was something we thought of at the informal interview.

\item Other than the risk of physical harm (some projects may not have any
appreciable risks of this form), list at least 2 other types of risk in
software products. Why are they important to consider?

Could be some performance risks and making sure that the performance of the software meets the goals/requirements for the project.
Another risk could be in terms of privacy. The application is capturing sensitive data and so its important on how the application handles this data.

\item What went well while writing this deliverable?

This deliverable was relatively painless and straightforward. We had already started thinking about testing plans
during our SRS deliverable, specifically section S.6. In this section we detailed a brief VnV plan, including system testing
and unit testing. This set up the basic outline for this deliverable, it being an extension of what we already wrote/thought about.

\item What pain points did you experience during this deliverable, and how
  did you resolve them?

One pain point we had during this deliverable was editing requirements from the SRS. When creating the traceability matrix, we noticed
that some of the requirements from the SRS document were overlapping or in the wrong spot. We held a meeting as a group to sort this out 
and reach a consensus on which requirements should stay, should be changed, or should be deleted.

\item What knowledge and skills will the team collectively need to acquire to
successfully complete the verification and validation of your project?
Examples of possible knowledge and skills include dynamic testing knowledge,
static testing knowledge, specific tool usage, Valgrind etc.  You should look to
identify at least one item for each team member.

In order to properly complete the verification and validation of our project, some skills will need to Be
acquired. Firstly, we will have to familiarize ourselves with cppunit, as majority of our testing knowledge from
previous courses is in Java or Python. Additionally, because of this, we will have to learn how to use GCov in order to
accurately figure out our code coverage. On the topic of code coverage, we will also need to brush up on our coverage 
definitions that were learnt in our testing course. Finally, we will need to implement linters to check our code on github
before merge.

\item For each of the knowledge areas and skills identified in the previous
question, what are at least two approaches to acquiring the knowledge or
mastering the skill?  Of the identified approaches, which will each team
member pursue, and why did they make this choice?

For these skills, there are a few ways to approach acquiring the knowledge. With new skills, utilizing Youtube and online tutorials
are a good way to quickly learn the basics and proper implementation of new techniques. For older skills, i.e. ones that we have learnt
previously but haven't used in a while, we can go back to old projects/lectures and relearn the information.

Matthew will find online tutorials to learn about cppunit. This is because he has no experience with creating automated testing in c++, and needs 
to start off by learning the basics.

Tarnveer will find online tutorials to learn about cppunit and c++ linters on Github. This is for the same reason as above; he has no experience
implementing testing in c++ or adding linters to Github for PRs.

Harman will go back to our old 3SO3 notes in order to relearn MC/DC coverage. This is because he had implemented MC/DC coverage and checks 
in that course, but in Java. Since he has implemented this before, he is familiar with the content and should be relatively painless to 
relearn the content.

Kyen will find online tutorials for learning about GCov. For similar reasons as Matthew and Tarnveer, this is because he has no prior experience with 
this specific coverage tool. 

\item What went well while writing this deliverable? 

  Everyone on the team was on track with their sections of the assignment and we were able to thick of better ways to break up some modules to make more sense.

\item What pain points did you experience during this deliverable, and how did you resolve them?

  Getting used to the new year and so it was a slow start trying to get back into the flow, but once we started working it came back.

\item Which of your design decisions stemmed from speaking to your client(s)or a proxy (e.g. your peers, stakeholders, potential users)? For those thatwere not, why, and where did they come from?

  Most of the module break up comes from talking to our client becuase they helped us focus on their vision for the project but making the inputs and specific variables were all done independently.

\item While creating the design doc, what parts of your other documents (e.g.
  requirements, hazard analysis, etc), it any, needed to be changed, and why?

  For now no real document had to be changed becuase the structure for this assignment was thought of before through the many client meets. This allowed for a strong structure.

\item What are the limitations of your solution?  Put another way, given
  unlimited resources, what could you do to make the project better? (LO\_ProbSolutions)

  With unlimited resources the ability to capture better imaging with the kinect would allow for a faster and more precise human detection algorithm. Maybe also being able to better maximize the human detection to better fit a humaniod shape.
\item Give a brief overview of other design solutions you considered.  What
  are the benefits and tradeoffs of those other designs compared with the chosen
  design?  From all the potential options, why did you select the documented design?
  (LO\_Explores)

  Other design implications would just involve taking a different approach to creating the algorithm. The issue with for example a solution that does not use hue or skin color is limiting our ability to full captalize on the fact that the sensor picks up RGB as well.

\end{enumerate}

\end{document}