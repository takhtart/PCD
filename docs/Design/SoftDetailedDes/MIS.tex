\documentclass[12pt, titlepage]{article}

\usepackage{amsmath, mathtools}

\usepackage[round]{natbib}
\usepackage{amsfonts}
\usepackage{amssymb}
\usepackage{graphicx}
\usepackage{colortbl}
\usepackage{xr}
\usepackage{hyperref}
\usepackage{longtable}
\usepackage{xfrac}
\usepackage{tabularx}
\usepackage{float}
\usepackage{siunitx}
\usepackage{booktabs}
\usepackage{multirow}
\usepackage[section]{placeins}
\usepackage{caption}
\usepackage{fullpage}

\hypersetup{
bookmarks=true,     % show bookmarks bar?
colorlinks=true,       % false: boxed links; true: colored links
linkcolor=red,          % color of internal links (change box color with linkbordercolor)
citecolor=blue,      % color of links to bibliography
filecolor=magenta,  % color of file links
urlcolor=cyan          % color of external links
}

\usepackage{array}

\externaldocument{../../SRS/SRS}

%% Comments

\usepackage{color}

\newif\ifcomments\commentstrue %displays comments
%\newif\ifcomments\commentsfalse %so that comments do not display

\ifcomments
\newcommand{\authornote}[3]{\textcolor{#1}{[#3 ---#2]}}
\newcommand{\todo}[1]{\textcolor{red}{[TODO: #1]}}
\else
\newcommand{\authornote}[3]{}
\newcommand{\todo}[1]{}
\fi

\newcommand{\wss}[1]{\authornote{blue}{SS}{#1}} 
\newcommand{\plt}[1]{\authornote{magenta}{TPLT}{#1}} %For explanation of the template
\newcommand{\an}[1]{\authornote{cyan}{Author}{#1}}

%% Common Parts

\newcommand{\progname}{PCD: Partially Covered Detection of Obscured People using Point Cloud Data} % PUT YOUR PROGRAM NAME HERE
\newcommand{\authname}{Team \#14, PCD
\\ Tarnveer Takhtar
\\ Matthew Bradbury
\\ Harman Bassi
\\ Kyen So} % AUTHOR NAMES                  

\usepackage{hyperref}
    \hypersetup{colorlinks=true, linkcolor=blue, citecolor=blue, filecolor=blue,
                urlcolor=blue, unicode=false}
    \urlstyle{same}

\usepackage{indentfirst}                              


\begin{document}

\title{Module Interface Specification for \progname{}}

\author{\authname}

\date{January 17, 2025}

\maketitle

\pagenumbering{roman}

\section{Revision History}

\begin{tabularx}{\textwidth}{p{3cm}p{2cm}X}
\toprule {\bf Date} & {\bf Version} & {\bf Notes}\\
\midrule
Jan 17, 2025 & 1.0 & Initial Draft\\
\bottomrule
\end{tabularx}

~\newpage

\section{Symbols, Abbreviations and Acronyms}

SRS Documentation can be found on \href{https://github.com/takhtart/PCD/blob/main/docs/SRS/SRS.pdf}{GitHub}.\\
\\
\renewcommand{\arraystretch}{1.2}
\begin{tabular}{l l} 
  \toprule		
  \textbf{symbol} & \textbf{description}\\
  \midrule 
  SRS & Software Requirements Specification\\
  PCD & Partially Covered Detection Program \\
  PCL & Point Cloud Library \\
  \bottomrule
\end{tabular}\\

\newpage

\tableofcontents

\newpage

\pagenumbering{arabic}

\section{Introduction}

The following document details the Module Interface Specifications for Partially Covered Detection (PCD) software. 
The sections in this document describes each module in our software and how each module interacts with each other. 
Additional information and documentation can be found in System Requirement Specifications (SRS).

Complementary documents include the System Requirement Specifications
and Module Guide.  The full documentation and implementation can be
found at \url{https://github.com/takhtart/PCD}. 

\section{Notation}

The structure of the MIS for modules comes from \citet{HoffmanAndStrooper1995},
with the addition that template modules have been adapted from
\cite{GhezziEtAl2003}.  The mathematical notation comes from Chapter 3 of
\citet{HoffmanAndStrooper1995}.  For instance, the symbol := is used for a
multiple assignment statement and conditional rules follow the form $(c_1
\Rightarrow r_1 | c_2 \Rightarrow r_2 | ... | c_n \Rightarrow r_n )$.

The following table summarizes the primitive data types used by \progname. 

\begin{center}
\renewcommand{\arraystretch}{1.2}
\noindent 
\begin{tabular}{l l p{7.5cm}} 
\toprule 
\textbf{Data Type} & \textbf{Notation} & \textbf{Description}\\ 
\midrule
character & char & a single symbol or digit\\
integer & $\mathbb{Z}$ & a number without a fractional component in (-$\infty$, $\infty$) \\
natural number & $\mathbb{N}$ & a number without a fractional component in [1, $\infty$) \\
real & $\mathbb{R}$ & any number in (-$\infty$, $\infty$)\\
PointXYZRGBA & $\mathbb{P}$ & point cloud data in the PCL Library\\
\bottomrule
\end{tabular} 
\end{center}

\noindent
The specification of PCD uses some derived data types: sequences, strings, and
tuples. Sequences are lists filled with elements of the same data type. Strings
are sequences of characters. Tuples contain a list of values, potentially of
different types. In addition, PCD uses functions, which
are defined by the data types of their inputs and outputs. Local functions are
described by giving their type signature followed by their specification.

\section{Module Decomposition}

The following table is taken directly from the Module Guide document for this project.

\begin{table}[h!]
\centering
\begin{tabular}{p{0.3\textwidth} p{0.6\textwidth}}
\toprule
\textbf{Level 1} & \textbf{Level 2}\\
\midrule

{Hardware-Hiding Module} & Kinect Stream \\
\midrule

\multirow{7}{0.3\textwidth}{Behaviour-Hiding Module} & Application Control\\
& Input Data Read\\
& Input Classifier\\
& Input Classifier Ranking\\
& Bounding Box Display\\
\midrule

\multirow{3}{0.3\textwidth}{Software Decision Module} & Point Cloud Data Structures\\
& Input Processing\\
& Command Line Interface\\
& Graphical User Interface\\
\bottomrule

\end{tabular}
\caption{Module Hierarchy}
\label{TblMH}
\end{table}

\newpage
~\newpage

\section{MIS of Kinect Stream} \label{ModuleKS} 

\subsection{Module}

kinect

\subsection{Uses}

\begin{itemize}
\item Input Data Read \ref{ModuleIDR}
\item Point Cloud Data Structure \ref{ModulePCDS}
\end{itemize}

\subsection{Syntax}

\subsubsection{Exported Constants}

None.

\subsubsection{Exported Access Programs}

\begin{center}
\begin{tabular}{p{2cm} p{4cm} p{5cm} p{2cm}}
\hline
\textbf{Name} & \textbf{In} & \textbf{Out} & \textbf{Exceptions} \\
\hline
kinect & video frame stream from kinect in BGRA format and depth frame stream
& - pcl::PointXYZ & - \\
 & & - pcl::PointXYZRGBA &  \\
 & & - pcl::PointXYZI &  \\
 & & - pcl::PointXYZRGB &  \\
\hline
\end{tabular}
\end{center}

\subsection{Semantics}

\subsubsection{State Variables}

None

\subsubsection{Environment Variables}

\begin{itemize}
  \item Size of the room (affects \# of points in point cloud)
  \item Lighting (affects \# of points and colour of the point cloud)
  \item Angle of the Kinect (affects layout of the points)
\end{itemize}

\subsubsection{Assumptions}

\begin{itemize}
  \item User selects live stream rather than offline view
  \item Kinect is connected and running without issue.
  \item Kinect has a clear and unobstructed view of the environment (lens are not covered)
\end{itemize}

\subsubsection{Access Routine Semantics}

\noindent kinect():
\begin{itemize}
  \item output: PointCloudT::ConstPtr input\_cloud
  \item Precondition: user calls live\_stream mode
  \item Postcondition: user terminates program or selects exit 
\end{itemize}

\subsubsection{Local Functions}

None

\newpage

\section{MIS of Application Control} \label{ModuleAC} 

\subsection{Module}

main

\subsection{Uses}

\begin{itemize}
  \item Input Data Read \ref{ModuleIDR}
  \item Input Classifier Module \ref{ModuleIC}
  \item Command Line Interface \ref{ModuleCLI}
  \item Graphical User Interface \ref{ModuleGUI}
  \item Bounding Box Display \ref{ModuleBBD}
\end{itemize}

\subsection{Syntax}

\subsubsection{Exported Constants}

None.

\subsubsection{Exported Access Programs}

\begin{center}
\begin{tabular}{p{2cm} p{4cm} p{4cm} p{2cm}}
\hline
\textbf{Name} & \textbf{In} & \textbf{Out} & \textbf{Exceptions} \\
\hline
main & None & None & - \\
\hline
\end{tabular}
\end{center}

\subsection{Semantics}

\subsubsection{State Variables}

PointcloudT::Ptr cloud 

\subsubsection{Environment Variables}

\begin{itemize}
\item Processing speed of device
\end{itemize}

\subsubsection{Assumptions}

Device has the processing power needed.

\subsubsection{Access Routine Semantics}

\noindent main():
\begin{itemize}
\item transition: connects the final filtered cloud (cloud after processing) to the GUI module to display the output.
\end{itemize}


\subsubsection{Local Functions}

None.

\newpage

\section{MIS of Input Data Read} \label{ModuleIDR} 

\subsection{Module}
reader
\subsection{Uses}

\subsection{Syntax}

\subsubsection{Exported Constants}

None.

\subsubsection{Exported Access Programs}

\begin{center}
\begin{tabular}{p{2cm} p{4cm} p{4cm} p{2cm}}
\hline
\textbf{Name} & \textbf{In} & \textbf{Out} & \textbf{Exceptions} \\
\hline
reader & std::cin user\_input  & None. & - \\
\hline
\end{tabular}
\end{center}

\subsection{Semantics}

\subsubsection{State Variables}

\begin{itemize}
  \item std::cin user\_input : Choice that the user made on what mode to run the program in.
\end{itemize}

\subsubsection{Environment Variables}


\subsubsection{Assumptions}

The user provides a valid input ($\mathbb{Z}$) corresponding with the correct option

\subsubsection{Access Routine Semantics}

\noindent main():
\begin{itemize}
  \item transition: Converts the input data into the data structure used by the Input Processing Module
\end{itemize}

\subsubsection{Local Functions}

None.

\newpage

\section{MIS of Input Classifier} \label{ModuleIC} 

\subsection{Module}

classify

\subsection{Uses}


\begin{itemize}
  \item Bounding Box Display \ref{ModuleBBD}
  \item Point Cloud Data Structures \ref{ModulePCDS}
\end{itemize}

\subsection{Syntax}

\subsubsection{Exported Constants}

None.

\subsubsection{Exported Access Programs}

\begin{center}
\begin{tabular}{p{2cm} p{5cm} p{4cm} p{2cm}}
\hline
\textbf{Name} & \textbf{In} & \textbf{Out} & \textbf{Exceptions} \\
\hline
classify & (PointCloudT::Ptr)cloud, (PointCloudT::Ptr) cloudfiltered & None. & - \\
\hline
\end{tabular}
\end{center}

\subsection{Semantics}

\subsubsection{State Variables}

\begin{itemize}
  \item *cloudFiltered = *personCloud : updated cloud value for the data set
  \item personCloud (PointCloudT::Ptr) : the cluster that is being identified as a human within the frame.
\end{itemize}

\subsubsection{Environment Variables}

None.

\subsubsection{Assumptions}

\begin{itemize}
  \item The cloud has been properly filtered to allow for a good reading to quickly identify the human on screen.
\end{itemize}

\subsubsection{Access Routine Semantics}

\noindent classify(cloud, cloudfiltered )():
transition: This module will take the the filtered values and filter down further to just identify the human within the frame and only leave the data points connected to that person.
Connects the filtered point cloud to the Application Control module
\subsubsection{Local Functions}

None.
\newpage

\section{MIS of Input Classifier Ranking} \label{ModuleICR} 

\subsection{Module}

ranking

\subsection{Uses}

\subsection{Syntax}

\subsubsection{Exported Constants}

None.

\subsubsection{Exported Access Programs}

\begin{center}
\begin{tabular}{p{2cm} p{4cm} p{4cm} p{2cm}}
\hline
\textbf{Name} & \textbf{In} & \textbf{Out} & \textbf{Exceptions} \\
\hline
ranking & dataPoint($\mathbb{P}$)  & None. & - \\
\hline
\end{tabular}
\end{center}

\subsection{Semantics}

\subsubsection{State Variables}

\begin{itemize}
  \item weights ($\mathbf{P}^{n}$): An array of size n containing the ordered weights of the pcd points
\end{itemize}

\subsubsection{Environment Variables}

None.

\subsubsection{Assumptions}

\begin{itemize}
  \item The input point cloud data is valid.
  \item The classification strategy is implemented correctly to be able to order the weights
\end{itemize}

\subsubsection{Access Routine Semantics}

\noindent ranking(dataPoint)():
\begin{itemize}
  \item transition: This module will take in the dataPoint and add it to the list and order it into the array. It connects the sorted array of ranking to the Input Classifier module.
\end{itemize}

\subsubsection{Local Functions}

None.
\newpage


\section{MIS of Bounding Box Display} \label{ModuleBBD} 

\subsection{Module}

boundingBox

\subsection{Uses}

\begin{itemize}
  \item Input Classifier Module
\end{itemize}

\subsection{Syntax}

\subsubsection{Exported Constants}

None.

\subsubsection{Exported Access Programs}

\begin{center}
\begin{tabular}{p{3cm} p{4cm} p{2cm} p{2cm}}
\hline
\textbf{Name} & \textbf{In} & \textbf{Out} & \textbf{Exceptions} \\
\hline
boundingBox & humancloud($\mathbb{P}$),
              minpt($\mathbb{P}$),
              maxpt($\mathbb{P}$) & None. & - \\
\hline
\end{tabular}
\end{center}

\subsection{Semantics}

\subsubsection{State Variables}

\begin{itemize}
  \item thickness ($\mathbf{R}$): An float with the thickness of the box
  \item $scale_factor$ ($\mathbf{R}$): Factor that adjusts the box size.
\end{itemize}

\subsubsection{Environment Variables}

None.

\subsubsection{Assumptions}

\begin{itemize}
  \item The input cloud points are valid to provide an accurate drawing of the box.
  \item The filtered cloud properly depicts a human.
\end{itemize}

\subsubsection{Access Routine Semantics}

\noindent boundingBox(humancloud,minpt,maxpt)():\\
transition: This module will take in inputed data points and add draw out a box using the max and min points provided. 
It connects the cloud and calculated bounding box to be presented at the GUI module.
\subsubsection{Local Functions}

None.
\newpage

\section{MIS of Point Cloud Data Structures} \label{ModulePCDS} 

\subsection{Module}

struct

\subsection{Uses}

\begin{itemize}
  \item Input Processing \ref{ModuleIP}
  \item Graphical User Interface \ref{ModuleGUI}
\end{itemize}

\subsection{Syntax}

\subsubsection{Exported Constants}

None.

\subsubsection{Exported Access Programs}

\begin{center}
\begin{tabular}{p{3cm} p{4cm} p{4cm} p{2cm}}
\hline
\textbf{Name} & \textbf{In} & \textbf{Out} & \textbf{Exceptions} \\
\hline
struct & None. & None. & - \\
\hline
\end{tabular}
\end{center}

\subsection{Semantics}

\subsubsection{State Variables}

\begin{itemize}
\item typedef pcl::PointXYZRGBA PointT : Data structure designed to store each individual point in a point cloud 
\item typedef pcl::PointCloud\textless PointT\textgreater PointCloudT : Data Structure designed to store an entire point cloud
\end{itemize}

\subsubsection{Environment Variables}

None.

\subsubsection{Assumptions}

The point cloud data from kinect stream is valid

\subsubsection{Access Routine Semantics}

\noindent struct():
\begin{itemize}
\item transition: This module is the point cloud data structure for storing point cloud data such as ones captured from the kinect and ones processed by our cloud processing algorithm. It connects generalized data types that abstract point cloud data to the Input Processing and GUI modules.
\end{itemize}


\subsubsection{Local Functions}

None.

\newpage

\section{MIS of Input Processing} \label{ModuleIP} 


\subsection{Module}

process\_cloudOCV

\subsection{Uses}

\begin{itemize}
\item Command Line Interface \ref{ModuleCLI}
\item Graphical User Interface \ref{ModuleGUI}
\end{itemize}

\subsection{Syntax}

\subsubsection{Exported Constants}

None.

\subsubsection{Exported Access Programs}

\begin{center}
\begin{tabular}{p{4cm} p{4cm} p{2cm} p{2cm}}
\hline
\textbf{Name} & \textbf{In} & \textbf{Out} & \textbf{Exceptions} \\
\hline
process\_cloudOCV & PointCloudT::Ptr cloud & None. & - \\
\hline
\end{tabular}
\end{center}

\subsection{Semantics}

\subsubsection{State Variables}

None.

\subsubsection{Environment Variables}

None.

\subsubsection{Assumptions}

Data captured by the kinect are correctly processed and stored in the point cloud data structure

\subsubsection{Access Routine Semantics}

\noindent process\_cloudOCV(cloud,cloud\_filtered):
\begin{itemize}
\item transition: it removes noise, perform plane removal, and skin point detection and transitions the detected skin points and filtered cloud to the Input Classifer module.
\end{itemize}


\subsubsection{Local Functions}

None.

\newpage

\section{MIS of Command Line Interface} \label{ModuleCLI} 

\subsection{Module}

cmd

\subsection{Uses}

\begin{itemize}
  \item Input Processing Module \ref{ModuleIP}
  \item Graphical User Interface \ref{ModuleGUI}
\end{itemize}



\subsection{Syntax}

\subsubsection{Exported Constants}

None.

\subsubsection{Exported Access Programs}

\begin{center}
\begin{tabular}{p{2cm} p{4cm} p{4cm} p{2cm}}
\hline
\textbf{Name} & \textbf{In} & \textbf{Out} & \textbf{Exceptions} \\
\hline
cmd & None. & None. & - \\
\hline
\end{tabular}
\end{center}

\subsection{Semantics}

\subsubsection{State Variables}

None.

\subsubsection{Environment Variables}

\begin{itemize}
  \item Keyboard (Used to select modes)
  \item Mouse (Interacts with command prompt)
\end{itemize}

\subsubsection{Assumptions}

\begin{itemize}
  \item Working keyboard and mouse is connected
  \item Kinect is properly setup and connected
  \item Proper file types are uploaded
\end{itemize}

\subsubsection{Access Routine Semantics}

\noindent cmd():
\begin{itemize}
\item transition: Provides the Data Read module with the user's option of offline versus live as well as the location of the downloaded .pcd file (for offline mode).
\end{itemize}

\subsubsection{Local Functions}

None.

\newpage

\section{MIS of Graphical User Interface} \label{ModuleGUI} 

\subsection{Module}

gui

\subsection{Uses}

None.

\subsection{Syntax}

\subsubsection{Exported Constants}

None.

\subsubsection{Exported Access Programs}

\begin{center}
\begin{tabular}{p{2cm} p{3cm} p{3cm} p{3cm}}
\hline
\textbf{Name} & \textbf{In} & \textbf{Out} & \textbf{Exceptions} \\
\hline
gui & PointcloudT::Ptr filtered\_cloud & Visulaized Point Cloud & - \\
\hline
\end{tabular}
\end{center}

\subsection{Semantics}

\subsubsection{State Variables}

\begin{itemize}
  \item std::thread visualizer\_thread (displaying the filtered cloud)
  \item std::thread visualizer\_thread2 (displaying the original cloud)
\end{itemize}

\subsubsection{Environment Variables}

\begin{itemize}
  \item Mouse (To move around within the visualized point cloud)
\end{itemize}

\subsubsection{Assumptions}

Point cloud was correctly processed and stored with the point cloud data structure


\subsubsection{Access Routine Semantics}

\noindent gui():
\begin{itemize}
\item output: Uses the visualizer to deploy a GUI which displays the 3D point clouds (filtered and original)
\end{itemize}

\subsubsection{Local Functions}

None.

\newpage

\bibliographystyle {plainnat}
\bibliography {../../../refs/References}

\newpage

\section{Appendix} \label{Appendix}

N/A

\newpage{}

\section*{Appendix --- Reflection}

\begin{enumerate}
  \item What went well while writing this deliverable? 

  Everyone on the team was on track with their sections of the assignment and we were able to thick of better ways to break up some modules to make more sense.

\item What pain points did you experience during this deliverable, and how did you resolve them?

  Getting used to the new year and so it was a slow start trying to get back into the flow, but once we started working it came back.

\item Which of your design decisions stemmed from speaking to your client(s)or a proxy (e.g. your peers, stakeholders, potential users)? For those thatwere not, why, and where did they come from?

  Most of the module break up comes from talking to our client becuase they helped us focus on their vision for the project but making the inputs and specific variables were all done independently.

\item While creating the design doc, what parts of your other documents (e.g.
  requirements, hazard analysis, etc), it any, needed to be changed, and why?

  For now no real document had to be changed becuase the structure for this assignment was thought of before through the many client meets. This allowed for a strong structure.

\item What are the limitations of your solution?  Put another way, given
  unlimited resources, what could you do to make the project better? (LO\_ProbSolutions)

  With unlimited resources the ability to capture better imaging with the kinect would allow for a faster and more precise human detection algorithm. Maybe also being able to better maximize the human detection to better fit a humaniod shape.
\item Give a brief overview of other design solutions you considered.  What
  are the benefits and tradeoffs of those other designs compared with the chosen
  design?  From all the potential options, why did you select the documented design?
  (LO\_Explores)

  Other design implications would just involve taking a different approach to creating the algorithm. The issue with for example a solution that does not use hue or skin color is limiting our ability to full captalize on the fact that the sensor picks up RGB as well.

\end{enumerate}

\end{document}