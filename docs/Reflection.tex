\begin{enumerate}
\item Why is it important to create a development plan prior to starting the
project?\\

It is important to create a development plan prior to starting the project as
it allows most of the heavy lifting behind project planning to be done before any
work has started. It allows for expectations and workflow to be clearly defined before
any issues arise. It also creates a document that can be referenced at other times 
to avoid confusion.

\item In your opinion, what are the advantages and disadvantages of using
CI/CD?\\

Employing CI/CD allows for better issue tracking and rollbacks. Utilizing CI/CD gives the
opportunity for teams to better track individual issues and commits, leading to increased 
awareness and visibility on workflow issues. It also allows for easier time rolling back
to a previous version in case something goes wrong. 
Some disadvantages are with the conceptual depth and speed. Ensuring a specific workflow 
and constant PR reviews can slow things down as contributors have to make sure that they
are following the workflow properly and have to wait for PR reviews (when necessary). Furthermore,
it is more effort to set up, both in the codebase and conceptually. The process has to be 
talked through and understood by all team members.

\item What disagreements did your group have in this deliverable, if any,
and how did you resolve them?\\

Our group mainly debated how to set up our GitHub workflows. Initially, we considered using individual branches for each issue, but we ultimately decided on a 
more streamlined approach with two revision branches and separate forks. This allows us to effectively manage pull requests for merging feature changes into the codebase. 
We reached this agreement after discussing the benefits of clarity and collaboration in our development process.

\item What went well while writing this deliverable? \\

This deliverable allowed for the group to be able to discuss standard goals vs stretched goals. Everyone contributed their ideas and we were able to come to a clear conclusion 
when it came to the goals and problem statement of the project. It also allowed us all to see where the project is headed and how we should properly prepare
 ourselves so that we can achieve our goals. 

\item What pain points did you experience during this deliverable, and how
did you resolve them?\\

The biggest pain point during this deliverable was being able to decide which goals were too ambitious and outside our design scope. Some goals such as the aspect of outlining the human 
in the environment were broken up. There was a deep discussion on how to properly decide the goals, but at the end the  group got a better understanding of the project.

\item How did you and your team adjust the scope of your goals to ensure
they are suitable for a Capstone project (not overly ambitious but also of
appropriate complexity for a senior design project)?\\

Because our project is presented by a professor, the team already had a pretty clear understanding of the goals that needed to be achieved for our project to be considered a success.
 The only issue was trying to ensure that the goals can be properly broken down so that a goal that seemed a bit ambitious could be broken into something that seems doable. For example, 
 the human detection is broken into the Minimal and viable product goal and then also extended into the stretch goal. This was an important discussion that the group had to ensure that we 
 deemed our goals to be doable.
\end{enumerate}

