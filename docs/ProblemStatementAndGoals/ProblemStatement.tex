\documentclass{article}

\usepackage{tabularx}
\usepackage{booktabs}

\title{Problem Statement and Goals\\\progname}

\author{\authname}

\date{}

%% Comments

\usepackage{color}

\newif\ifcomments\commentstrue %displays comments
%\newif\ifcomments\commentsfalse %so that comments do not display

\ifcomments
\newcommand{\authornote}[3]{\textcolor{#1}{[#3 ---#2]}}
\newcommand{\todo}[1]{\textcolor{red}{[TODO: #1]}}
\else
\newcommand{\authornote}[3]{}
\newcommand{\todo}[1]{}
\fi

\newcommand{\wss}[1]{\authornote{blue}{SS}{#1}} 
\newcommand{\plt}[1]{\authornote{magenta}{TPLT}{#1}} %For explanation of the template
\newcommand{\an}[1]{\authornote{cyan}{Author}{#1}}

%% Common Parts

\newcommand{\progname}{PCD: Partially Covered Detection of Obscured People using Point Cloud Data} % PUT YOUR PROGRAM NAME HERE
\newcommand{\authname}{Team \#14, PCD
\\ Tarnveer Takhtar
\\ Matthew Bradbury
\\ Harman Bassi
\\ Kyen So} % AUTHOR NAMES                  

\usepackage{hyperref}
    \hypersetup{colorlinks=true, linkcolor=blue, citecolor=blue, filecolor=blue,
                urlcolor=blue, unicode=false}
    \urlstyle{same}

\usepackage{indentfirst}                              


\begin{document}

\maketitle

\begin{table}[hp]
\caption{Revision History} \label{TblRevisionHistory}
\begin{tabularx}{\textwidth}{llX}
\toprule
\textbf{Date} & \textbf{Developer(s)} & \textbf{Change}\\
\midrule
Date1 & Name(s) & Description of changes\\
Date2 & Name(s) & Description of changes\\
... & ... & ...\\
\bottomrule
\end{tabularx}
\end{table}

\section{Problem Statement}

\wss{You should check your problem statement with the
\href{https://github.com/smiths/capTemplate/blob/main/docs/Checklists/ProbState-Checklist.pdf}
{problem statement checklist}.} 

\wss{You can change the section headings, as long as you include the required
information.}

\subsection{Problem}

\subsection{Inputs and Outputs}

\wss{Characterize the problem in terms of ``high level'' inputs and outputs.  
Use abstraction so that you can avoid details.}

\subsection{Stakeholders}

\subsection{Environment}

\wss{Hardware and software environment}

\section{Goals}

\section{Stretch Goals}

\section{Challenge Level and Extras}

\wss{State your expected challenge level (advanced, general or basic).  The
challenge can come through the required domain knowledge, the implementation or
something else.  Usually the greater the novelty of a project the greater its
challenge level.  You should include your rationale for the selected level.
Approval of the level will be part of the discussion with the instructor for
approving the project.  The challenge level, with the approval (or request) of
the instructor, can be modified over the course of the term.}

\wss{Teams may wish to include extras as either potential bonus grades, or to
make up for a less advanced challenge level.  Potential extras include usability
testing, code walkthroughs, user documentation, formal proof, GenderMag
personas, Design Thinking, etc.  Normally the maximum number of extras will be
two.  Approval of the extras will be part of the discussion with the instructor
for approving the project.  The extras, with the approval (or request) of the
instructor, can be modified over the course of the term.}

\newpage{}

\section*{Appendix --- Reflection}

\wss{Not required for CAS 741}

\begin{enumerate}
\item Why is it important to create a development plan prior to starting the
project?\\

It is important to create a development plan prior to starting the project as
it allows most of the heavy lifting behind project planning to be done before any
work has started. It allows for expectations and workflow to be clearly defined before
any issues arise. It also creates a document that can be referenced at other times 
to avoid confusion.

\item In your opinion, what are the advantages and disadvantages of using
CI/CD?\\

Employing CI/CD allows for better issue tracking and rollbacks. Utilizing CI/CD gives the
opportunity for teams to better track individual issues and commits, leading to increased 
awareness and visibility on workflow issues. It also allows for easier time rolling back
to a previous version in case something goes wrong. 
Some disadvantages are with the conceptual depth and speed. Ensuring a specific workflow 
and constant PR reviews can slow things down as contributors have to make sure that they
are following the workflow properly and have to wait for PR reviews (when necessary). Furthermore,
it is more effort to set up, both in the codebase and conceptually. The process has to be 
talked through and understood by all team members.

\item What disagreements did your group have in this deliverable, if any,
and how did you resolve them?\\

Our group mainly debated how to set up our GitHub workflows. Initially, we considered using individual branches for each issue, but we ultimately decided on a 
more streamlined approach with two revision branches and separate forks. This allows us to effectively manage pull requests for merging feature changes into the codebase. 
We reached this agreement after discussing the benefits of clarity and collaboration in our development process.

\item What went well while writing this deliverable? \\

This deliverable allowed for the group to be able to discuss standard goals vs stretched goals. Everyone contributed their ideas and we were able to come to a clear conclusion 
when it came to the goals and problem statement of the project. It also allowed us all to see where the project is headed and how we should properly prepare
 ourselves so that we can achieve our goals. 

\item What pain points did you experience during this deliverable, and how
did you resolve them?\\

The biggest pain point during this deliverable was being able to decide which goals were too ambitious and outside our design scope. Some goals such as the aspect of outlining the human 
in the environment were broken up. There was a deep discussion on how to properly decide the goals, but at the end the  group got a better understanding of the project.

\item How did you and your team adjust the scope of your goals to ensure
they are suitable for a Capstone project (not overly ambitious but also of
appropriate complexity for a senior design project)?\\

Because our project is presented by a professor, the team already had a pretty clear understanding of the goals that needed to be achieved for our project to be considered a success.
 The only issue was trying to ensure that the goals can be properly broken down so that a goal that seemed a bit ambitious could be broken into something that seems doable. For example, 
 the human detection is broken into the Minimal and viable product goal and then also extended into the stretch goal. This was an important discussion that the group had to ensure that we 
 deemed our goals to be doable.

\item What knowledge and skills will the team collectively need to acquire to successfully complete this capstone project?  Examples of possible knowledge to acquire include domain specific knowledge 
from the domain of your application, or software engineering knowledge, mechatronics knowledge or computer science knowledge.  Skills may be related to technology, or writing, or presentation, 
or team management, etc.  You should look to identify at least one item for each team member.

Every member of the group needs to acquire knowledge on Computer Vision and understanding how pcd files operate. Overall these are big topics and so the basics should be acquired by every member, but 
some specifics would be broken down to different parts for each member. Tarnveer and Matthew would be assigned to understanding how to track people on screen and map them on the screen. Harman and Kyen 
would be working on reading in the pcd files and understanding the PCL. The PCL will explore on aspects of boxing the points on screen. Tarnveer would also be responsible for understanding
the real time coding aspect in c++. Matthew would also be assigned to acquire knowledge on improving the human outline based on better data. Harman will be assigned to understanding how to cancel out noise
from the data set. Kyen will have to understand how to find the person based off the pcd and understand how to properly read the files.

\item For each of the knowledge areas and skills identified in the previous question, what are at least two approaches to acquiring the knowledge or mastering the skill?  Of the identified approaches, which
will each team member pursue, and why did they make this choice?

One approach for acquiring the knowledge would be to use the provided documentation for the libraries (PCL and OpenCV). This would provide a good fundamental understanding for the important aspects of the project. 

Another approach would be to just watch videos on the specific topics and try to understand from there. This would be able to provide a more visual explanations for the topics.

Tarnveer: use documentation and videos
Matthew: use documentation and videos 
Kyen: use documentation
Harman: use documentation and videos

\item What went well while writing this deliverable? 

The deliverable was straightforward. We had a rough idea of the main hazards within our project and 
tried to make sure that we covered the main scope. The document writing was split between all of us.
The document is pretty straightforward and we as a group were able to talk over the different sections
and divide up the work.

\item What pain points did you experience during this deliverable, and how
did you resolve them?

The biggest pain point was probably discussing what would be some assumptions we had to make, but we were able to come to an agreement by communicating our points of why or why not.

\item Which of your listed risks had your team thought of before this
deliverable, and which did you think of while doing this deliverable? For
the latter ones (ones you thought of while doing the Hazard Analysis), how
did they come about?

All the risks were mainly thought of before the deliverable. We knew that we needed to ensure that the 
offline file is the correct format and that the system needs to make sure it is working with a Kinect sensor and not something else.

The privacy risk was something we thought of at the informal interview.

\item Other than the risk of physical harm (some projects may not have any
appreciable risks of this form), list at least 2 other types of risk in
software products. Why are they important to consider?

Could be some performance risks and making sure that the performance of the software meets the goals/requirements for the project.
Another risk could be in terms of privacy. The application is capturing sensitive data and so its important on how the application handles this data.

\item What went well while writing this deliverable?

This deliverable was relatively painless and straightforward. We had already started thinking about testing plans
during our SRS deliverable, specifically section S.6. In this section we detailed a brief VnV plan, including system testing
and unit testing. This set up the basic outline for this deliverable, it being an extension of what we already wrote/thought about.

\item What pain points did you experience during this deliverable, and how
  did you resolve them?

One pain point we had during this deliverable was editing requirements from the SRS. When creating the traceability matrix, we noticed
that some of the requirements from the SRS document were overlapping or in the wrong spot. We held a meeting as a group to sort this out 
and reach a consensus on which requirements should stay, should be changed, or should be deleted.

\item What knowledge and skills will the team collectively need to acquire to
successfully complete the verification and validation of your project?
Examples of possible knowledge and skills include dynamic testing knowledge,
static testing knowledge, specific tool usage, Valgrind etc.  You should look to
identify at least one item for each team member.

In order to properly complete the verification and validation of our project, some skills will need to Be
acquired. Firstly, we will have to familiarize ourselves with cppunit, as majority of our testing knowledge from
previous courses is in Java or Python. Additionally, because of this, we will have to learn how to use GCov in order to
accurately figure out our code coverage. On the topic of code coverage, we will also need to brush up on our coverage 
definitions that were learnt in our testing course. Finally, we will need to implement linters to check our code on github
before merge.

\item For each of the knowledge areas and skills identified in the previous
question, what are at least two approaches to acquiring the knowledge or
mastering the skill?  Of the identified approaches, which will each team
member pursue, and why did they make this choice?

For these skills, there are a few ways to approach acquiring the knowledge. With new skills, utilizing Youtube and online tutorials
are a good way to quickly learn the basics and proper implementation of new techniques. For older skills, i.e. ones that we have learnt
previously but haven't used in a while, we can go back to old projects/lectures and relearn the information.

Matthew will find online tutorials to learn about cppunit. This is because he has no experience with creating automated testing in c++, and needs 
to start off by learning the basics.

Tarnveer will find online tutorials to learn about cppunit and c++ linters on Github. This is for the same reason as above; he has no experience
implementing testing in c++ or adding linters to Github for PRs.

Harman will go back to our old 3SO3 notes in order to relearn MC/DC coverage. This is because he had implemented MC/DC coverage and checks 
in that course, but in Java. Since he has implemented this before, he is familiar with the content and should be relatively painless to 
relearn the content.

Kyen will find online tutorials for learning about GCov. For similar reasons as Matthew and Tarnveer, this is because he has no prior experience with 
this specific coverage tool. 

\item What went well while writing this deliverable? 

  Everyone on the team was on track with their sections of the assignment and we were able to thick of better ways to break up some modules to make more sense.

\item What pain points did you experience during this deliverable, and how did you resolve them?

  Getting used to the new year and so it was a slow start trying to get back into the flow, but once we started working it came back.

\item Which of your design decisions stemmed from speaking to your client(s)or a proxy (e.g. your peers, stakeholders, potential users)? For those thatwere not, why, and where did they come from?

  Most of the module break up comes from talking to our client becuase they helped us focus on their vision for the project but making the inputs and specific variables were all done independently.

\item While creating the design doc, what parts of your other documents (e.g.
  requirements, hazard analysis, etc), it any, needed to be changed, and why?

  For now no real document had to be changed becuase the structure for this assignment was thought of before through the many client meets. This allowed for a strong structure.

\item What are the limitations of your solution?  Put another way, given
  unlimited resources, what could you do to make the project better? (LO\_ProbSolutions)

  With unlimited resources the ability to capture better imaging with the kinect would allow for a faster and more precise human detection algorithm. Maybe also being able to better maximize the human detection to better fit a humaniod shape.
\item Give a brief overview of other design solutions you considered.  What
  are the benefits and tradeoffs of those other designs compared with the chosen
  design?  From all the potential options, why did you select the documented design?
  (LO\_Explores)

  Other design implications would just involve taking a different approach to creating the algorithm. The issue with for example a solution that does not use hue or skin color is limiting our ability to full captalize on the fact that the sensor picks up RGB as well.

\end{enumerate}

\begin{enumerate}
    \item What went well while writing this deliverable? 
    \item What pain points did you experience during this deliverable, and how
    did you resolve them?
    \item How did you and your team adjust the scope of your goals to ensure
    they are suitable for a Capstone project (not overly ambitious but also of
    appropriate complexity for a senior design project)?
\end{enumerate}  

\end{document}