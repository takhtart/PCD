\documentclass{article}

\usepackage{booktabs}
\usepackage{tabularx}

\title{Development Plan\\\progname}

\author{\authname}

\date{}

%% Comments

\usepackage{color}

\newif\ifcomments\commentstrue %displays comments
%\newif\ifcomments\commentsfalse %so that comments do not display

\ifcomments
\newcommand{\authornote}[3]{\textcolor{#1}{[#3 ---#2]}}
\newcommand{\todo}[1]{\textcolor{red}{[TODO: #1]}}
\else
\newcommand{\authornote}[3]{}
\newcommand{\todo}[1]{}
\fi

\newcommand{\wss}[1]{\authornote{blue}{SS}{#1}} 
\newcommand{\plt}[1]{\authornote{magenta}{TPLT}{#1}} %For explanation of the template
\newcommand{\an}[1]{\authornote{cyan}{Author}{#1}}

%% Common Parts

\newcommand{\progname}{PCD: Partially Covered Detection of Obscured People using Point Cloud Data} % PUT YOUR PROGRAM NAME HERE
\newcommand{\authname}{Team \#14, PCD
\\ Tarnveer Takhtar
\\ Matthew Bradbury
\\ Harman Bassi
\\ Kyen So} % AUTHOR NAMES                  

\usepackage{hyperref}
    \hypersetup{colorlinks=true, linkcolor=blue, citecolor=blue, filecolor=blue,
                urlcolor=blue, unicode=false}
    \urlstyle{same}

\usepackage{indentfirst}                              


\begin{document}

\maketitle

\begin{table}[hp]
\caption{Revision History} \label{TblRevisionHistory}
\begin{tabularx}{\textwidth}{llX}
\toprule
\textbf{Date} & \textbf{Developer(s)} & \textbf{Change}\\
\midrule
September 24, 2024 & Tarnveer Takhtar, Matthew Bradbury & Initial Draft\\
March 26, 2025 & Harman Bassi & \href{https://github.com/takhtart/PCD/issues/74}{Revision 1}\\
March 26, 2025 & Matthew Bradbury & \href{https://github.com/takhtart/PCD/issues/75}{Revision 1}\\
March 26, 2025 & Matthew Bradbury & \href{https://github.com/takhtart/PCD/issues/76}{Revision 1}\\
March 26, 2025 & Harman Bassi & \href{https://github.com/takhtart/PCD/issues/77}{Revision 1}\\
March 26, 2025 & Matthew Bradbury & \href{https://github.com/takhtart/PCD/issues/78}{Revision 1}\\
March 26, 2025 & Harman Bassi & \href{https://github.com/takhtart/PCD/issues/79}{Revision 1}\\
\bottomrule
\end{tabularx}
\end{table}

\newpage{}

This report outlines the key aspects of our project, including our structured 
team meeting and communication plans. It details the roles and responsibilities of team members, 
our workflow plan, and our project decomposition and scheduling strategy. Additionally, it covers our proof of concept demonstration plan, the expected 
technology stack, and the coding standards we will adhere to. The appendices include reflections and our
team charter, which highlights our external goals, attendance expectations, and accountability measures.

\section{Confidential Information?}

No confidential information present at this time.

\section{IP to Protect}

No IP to protect at this time.

\section{Copyright License}
Our project will be using the MIT License.

\section{Team Meeting Plan}

We will be meeting once a week, apart from lecture times, in a physical location on 
McMaster campus. Alternatively, the meeting can be held virtually on Discord or Microsoft Teams incase of
unforeseen circumstances.\\

We will provide our advisor with updates bi-weekly, scheduling an meeting
for larger updates and sending an email for smaller updates. Meetings will
be held virtually unless for specific reasons. Virtual meetings will be hosted on
Microsoft Teams.\\

For each meeting the scheduler of the meeting will be the chair. 
For recurring weekly meetings, the team lead will be the chair. 
Agenda will be prepared by the chair of the meeting.\\

The structure of the meetings will follow a simple outline. The meeting will begin with the discussion of last weeks meeting actions.
It will then cover any new issues that have risen since our last meeting, followed by a discussion of what is need to be completed this week. The meeting will end with a quick summary of next weeks actions and how the work will be divided for the following week.\\

The advisor meetings will follow a similar structure, but are there to a dedicated time to answer the advisors questions and ask for help for any incompleted tasks. They will also end with a quick post of the meeting minutes from the discussion. 
\section{Team Communication Plan}

Github issues, and a regularly updated Kanban board using those issues, will be used as a means of asynchronous communication 
between members and our Advisor outside of regularly scheduled meetings hosted on Microsoft Teams. 
Discord will be used for general communications between team members. Our Advisor will be contacted via Email or Microsoft Teams.

\section{Team Member Roles}
\noindent Team Lead:\\
Tarnveer Takhtar\\
This role entails communication with instructors/supervisors and keeping team members on task during meetings.\\
\\
Approver:\\
Matthew Bradbury\\
This role entails reviewing the created pull requests as a final check before approving them for merge.\\
\\
Reviewer(s):\\
Harman Bassi\\
Kyen So\\
This role entails reviewing documentation/coding changes prior to creating pull requests.\\
\\
Administrator:\\
Kyen So\\
This role entails keeping notes on meeting points discussed and issuing action items to corresponding team members.\\
\\
Coordinator:\\
Harman Bassi\\
This role entails planning out meeting schedules and ensuring no scheduling conflicts for team members.\\
\\
All members will work together to be responsible with breaking down responsibilities in terms of deliverables. Furthermore, these roles are somewhat flexible based on the availability of each team member; some member may ask another to cover their responsibilities if needed.
Additionally, all members will be responsible for preparing the necessary background needed for this project. As mentioned in the problem statement, this is an advanced project partially due to the lack of experience each member has with computer vision. Therefore, it is the responsibility of each member to ensure that the necessary background is fulfilled.
\section{Workflow Plan}

We will be using a branch for each Revision, and separate branches for each
major code features. Pull requests will be reserved for merging feature branches,
while documentation changes will be manually reviewed and edited.\\

Github issues will be created for any tasks as per the course outline, as well as being created also for code features.
Issue priority and classification will be designated by both deadline, and if it is a pre-requisite of other tasks.
Issues will be assigned accordingly as discussed within our meetings. 
If any discrepancies are found within the documentation, an issue will be made, and assigned. 
These issues will be formatted according to its type, either creation or revision. Documentation/code creation issues will provide the documentation, a brief description of the expectations, and the due date. 
Revision issues will provide the changed section, the reason, and a description of what was changed.\\

GitHub Actions will be used to handle CI/CD, triggered each time a pull request occurs for the codebase of the project. 
A predefined linter will go over the code ensuring consistency with the chosen linter's standard. 
Additional workflow tests may also be present to ensure compatibility.


\section{Project Decomposition and Scheduling}

Our Github project will consist of a Kanban board, highlighting the tasks that still need to be done, in progress or completed.\\
\\
\noindent Link to Kanban Board:
\noindent\href{https://github.com/users/takhtart/projects/3}{https://github.com/users/takhtart/projects/3}\\
\\
The Kanban board will contain a high-level overview of the tasks that need to be done, 
with priority based on deadlines provided by the course outline.

\section{Proof of Concept Demonstration Plan}

\subsection*{Significant Risks}
\begin{enumerate}
\item Project Complexity: The primary risk is the potential complexity of the project, which may be greater than initially anticipated. The main concern that arises is the ability to accurately detect a human. 
The added complexity for this comes from the restriction on the use of trained models to help with human detection.
Our team welcomes the challenges associated with this project however, we must understand the risks of taking on a project where we have no experience in the specific domain.

\item Kinect Sensor Performance: Another critical concern is the performance of the provided Kinect sensor, both during the development phase and the demonstration. 
Ensuring the sensor functions optimally and produces a quality real-time point cloud image is crucial for the project's success.

\end{enumerate}

Overall, The performance and reliability of the Kinect sensor and the implementation complexity of the detecting people when only partially visible are our main concerns regarding this project.

\subsection*{Demonstration Plan}

To demonstrate that these risks can be overcome, we plan on showcasing the system's ability to reliably identify people \textbf{not} obscured by any objects (base case). 
This will be a key achievement for our team, instilling confidence within our team that we can take on the primary challenge of identifying people when obscured.
This process will also present the Kinect sensor's performance during both development and the demo, providing more first hand knowledge on the sensor's benefits and drawbacks we should be aware of.

\section{Expected Technology}

We plan to use C++ as our primary programming language. 
C++ is well-suited for data processing and is low-level, which we will likely need for our project to keep processing time fast.
We anticipate utilizing libraries such as OpenCV and PCL (point cloud library) for handling point cloud data with a Kinect as our input.\\

Additionally, to maintain code quality, we will employ tools like CUDA for linting and GTest for unit testing. 
For code coverage we plan to use OpenCppCoverage to ensure thorough testing.
For CI, we will likely use GitHub Actions to automate testing and deployment.
Collaboration and version control will be conducted through Git and GitHub. Github Projects will be used to manage opened issue tickets, and be our primary tracker for project progression.
Performance monitoring tools such as Valgrind may be considered for optimizing critical code sections (optimizing real-time detection). 

\section{Coding Standard}
The coding standard we will adopt will be C++ 20. 

\newpage{}

\section*{Appendix --- Reflection}

\begin{enumerate}
    \item Why is it important to create a development plan prior to starting the
project?\\

It is important to create a development plan prior to starting the project as
it allows most of the heavy lifting behind project planning to be done before any
work has started. It allows for expectations and workflow to be clearly defined before
any issues arise. It also creates a document that can be referenced at other times 
to avoid confusion.

\item In your opinion, what are the advantages and disadvantages of using
CI/CD?\\

Employing CI/CD allows for better issue tracking and rollbacks. Utilizing CI/CD gives the
opportunity for teams to better track individual issues and commits, leading to increased 
awareness and visibility on workflow issues. It also allows for easier time rolling back
to a previous version in case something goes wrong. 
Some disadvantages are with the conceptual depth and speed. Ensuring a specific workflow 
and constant PR reviews can slow things down as contributors have to make sure that they
are following the workflow properly and have to wait for PR reviews (when necessary). Furthermore,
it is more effort to set up, both in the codebase and conceptually. The process has to be 
talked through and understood by all team members.

\item What disagreements did your group have in this deliverable, if any,
and how did you resolve them?\\

Our group mainly debated how to set up our GitHub workflows. Initially, we considered using individual branches for each issue, but we ultimately decided on a 
more streamlined approach with two revision branches and separate forks. This allows us to effectively manage pull requests for merging feature changes into the codebase. 
We reached this agreement after discussing the benefits of clarity and collaboration in our development process.

\end{enumerate}

\newpage{}

\section*{Appendix --- Team Charter}

\subsection*{External Goals}

The team's external goals for this project are to create an impressive project
to discuss during interviews. Additionally, we aim to showcase a challenging
and noteworthy project at the capstone convention. We also want to develop new skills 
and refine our existing ones throughout this project.

\subsection*{Attendance}

\subsubsection*{Expectations}

The expectation is that members clearly communicate their availability,
and give at least 24 hour notice when not being able to attend meetings.
If a member needs to skip/leave early, they are expected to communicate 
what work they will complete prior to the next meeting and ensure that
they are up to date on any project changes. While there is no \% of
meetings that must be attended, we will be using group consensus to
determine if one particular group member is missing a worrying number
of meetings.

\subsubsection*{Acceptable Excuse}

An acceptable excuse for missing a meeting or deadline would be any truly
unforeseen/unavoidable engagement which requires immediate attention or any engagement 
that is properly communicated beforehand. The second type of excuse will not be 
permitted if the rest of the group feels that an individual member has missed too 
many meetings/deadlines.

\subsubsection*{In Case of Emergency}

In case of emergency, the individual should convey exactly what they can and cannot accomplish to
the team, and ensure that they complete an adequate amount of work at a later date 
for whoever ends up covering.

\subsection*{Accountability and Teamwork}

\subsubsection*{Quality}

Team members are expected to adequately prepare for team meetings. This means completing 
agreed upon work prior to meeting times, barring a reasonable excuse. For supervisor/TA
meetings, members are expected to adequately understand the reason for the meeting enough
to contribute unique ideas/questions to the conversation.\\ 

Regarding quality of work, members are expected to complete work at a reasonable pace, allowing room 
for adequate review to take place prior to deadlines. The work should also be done at a quality
similar to the others, i.e. everyone is contributing at a quality similar to the person of highest
engagement/quality level.

\subsubsection*{Attitude}

The team's expectations are that each member completes work with a level of engagement
relatively similar to other team members. When providing ideas/interactions, it is 
expected that each member is treated with respect.

\subsubsection*{Stay on Track}

We will use Github projects to ensure that we are on track. Each task to work on
will be stored in an issue, which will have up to date deadlines attached to them, 
as well as the assignee(s). This will give us the ability to check up and see if 
tasks are being worked on/reviewed at a timely pace. On top of issues, we will 
be looking at commits and team meeting attendance to gauge contributions.\\

In the event that members aren't contributing their fair share, that member will have
to purchase snacks for the next team meeting. We will also have a group meeting with 
them to discuss steps to improve and expectations. If the behaviour continues further, 
we will escalate the issue to a TA/professor.

\subsubsection*{Team Building}

Team cohesion will be built through friendly conversation/banter and engaging
in fun/relaxing group activities.

\subsubsection*{Decision Making} 

We will make decisions based on group consensus, and if the vote isn't unanimous we will try to solve the dispute
via discussion until the decision is unanimous. If disagrees propagate further, we will escalate to asking our supervisor
or TA for advice on how to proceed.

\end{document}