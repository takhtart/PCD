\documentclass[12pt, titlepage]{article}

\usepackage{booktabs}
\usepackage{tabularx}
\usepackage{hyperref}
\hypersetup{
    colorlinks,
    citecolor=blue,
    filecolor=black,
    linkcolor=red,
    urlcolor=blue
}
\usepackage[round]{natbib}

\input{../Comments}
%% Common Parts

\newcommand{\progname}{PCD: Partially Covered Detection of Obscured People using Point Cloud Data} % PUT YOUR PROGRAM NAME HERE
\newcommand{\authname}{Team \#14, PCD
\\ Tarnveer Takhtar
\\ Matthew Bradbury
\\ Harman Bassi
\\ Kyen So} % AUTHOR NAMES                  

\usepackage{hyperref}
    \hypersetup{colorlinks=true, linkcolor=blue, citecolor=blue, filecolor=blue,
                urlcolor=blue, unicode=false}
    \urlstyle{same}
                                


\begin{document}

\title{System Verification and Validation Plan for \progname{}} 
\author{\authname}
\date{\today}
	
\maketitle

\pagenumbering{roman}

\section*{Revision History}

\begin{tabularx}{\textwidth}{p{3cm}p{2cm}X}
\toprule {\bf Date} & {\bf Version} & {\bf Notes}\\
\midrule
Date 1 & 1.0 & Notes\\
Date 2 & 1.1 & Notes\\
\bottomrule
\end{tabularx}

~\\
\wss{The intention of the VnV plan is to increase confidence in the software.
However, this does not mean listing every verification and validation technique
that has ever been devised.  The VnV plan should also be a \textbf{feasible}
plan. Execution of the plan should be possible with the time and team available.
If the full plan cannot be completed during the time available, it can either be
modified to ``fake it'', or a better solution is to add a section describing
what work has been completed and what work is still planned for the future.}

\wss{The VnV plan is typically started after the requirements stage, but before
the design stage.  This means that the sections related to unit testing cannot
initially be completed.  The sections will be filled in after the design stage
is complete.  the final version of the VnV plan should have all sections filled
in.}

\newpage

\tableofcontents

\listoftables
\wss{Remove this section if it isn't needed}

\listoffigures
\wss{Remove this section if it isn't needed}

\newpage

\section{Symbols, Abbreviations, and Acronyms}

\renewcommand{\arraystretch}{1.2}
\begin{tabular}{l l} 
  \toprule		
  \textbf{symbol} & \textbf{description}\\
  \midrule 
  T & Test\\
  \bottomrule
\end{tabular}\\

\wss{symbols, abbreviations, or acronyms --- you can simply reference the SRS
  \citep{SRS} tables, if appropriate}

\wss{Remove this section if it isn't needed}

\newpage

\pagenumbering{arabic}

This document ... \wss{provide an introductory blurb and roadmap of the
  Verification and Validation plan}

\section{General Information}

\subsection{Summary}

\wss{Say what software is being tested.  Give its name and a brief overview of
  its general functions.}

\subsection{Objectives}

\wss{State what is intended to be accomplished.  The objective will be around
  the qualities that are most important for your project.  You might have
  something like: ``build confidence in the software correctness,''
  ``demonstrate adequate usability.'' etc.  You won't list all of the qualities,
  just those that are most important.}

\wss{You should also list the objectives that are out of scope.  You don't have 
the resources to do everything, so what will you be leaving out.  For instance, 
if you are not going to verify the quality of usability, state this.  It is also 
worthwhile to justify why the objectives are left out.}

\wss{The objectives are important because they highlight that you are aware of 
limitations in your resources for verification and validation.  You can't do everything, 
so what are you going to prioritize?  As an example, if your system depends on an 
external library, you can explicitly state that you will assume that external library 
has already been verified by its implementation team.}

\subsection{Challenge Level and Extras}

\wss{State the challenge level (advanced, general, basic) for your project.
Your challenge level should exactly match what is included in your problem
statement.  This should be the challenge level agreed on between you and the
course instructor.  You can use a pull request to update your challenge level
(in TeamComposition.csv or Repos.csv) if your plan changes as a result of the
VnV planning exercise.}

\wss{Summarize the extras (if any) that were tackled by this project.  Extras
can include usability testing, code walkthroughs, user documentation, formal
proof, GenderMag personas, Design Thinking, etc.  Extras should have already
been approved by the course instructor as included in your problem statement.
You can use a pull request to update your extras (in TeamComposition.csv or
Repos.csv) if your plan changes as a result of the VnV planning exercise.}

\subsection{Relevant Documentation}

\wss{Reference relevant documentation.  This will definitely include your SRS
  and your other project documents (design documents, like MG, MIS, etc).  You
  can include these even before they are written, since by the time the project
  is done, they will be written.  You can create BibTeX entries for your
  documents and within those entries include a hyperlink to the documents.}

\citet{SRS}

\wss{Don't just list the other documents.  You should explain why they are relevant and 
how they relate to your VnV efforts.}

\section{Plan}

\wss{Introduce this section.  You can provide a roadmap of the sections to
  come.}

This section describes the overall plan for the verification and validation of our system. It includes the work breakdown 
of each member of the verification and validation team. This section also outlines the plans for the verification of 
our SRS, Design, and VnV. Furthermore, it details the plans for the implementation of these verification strategies as well as 
the implementation of the testing tools and the software validation plan.

This section describes the overall plan for the verification and validation of our system. It includes the work breakdown 
of each member of the verification and validation team. This section also outlines the plans for the verification of 
our SRS, Design, and VnV. Furthermore, it details the plans for the implementation of these verification strategies as well as 
the implementation of the testing tools and the software validation plan.

\subsection{Verification and Validation Team}

\wss{Your teammates.  Maybe your supervisor.
  You should do more than list names.  You should say what each person's role is
  for the project's verification.  A table is a good way to summarize this information.}

  \begin{table}
  \caption{Verification and Validation Team Members Table}
  \centering
  \begin{tabular}{|l|p{1.8in}|p{2.5in}|}
  \hline
  \textbf{Name}            & \textbf{Role(s)}                                       & \textbf{Responsibilities}                                                                                                                                             \\ \hline
  Harman Bassi         & Lead test developer, Test developer, Manual tester               & Lead the test development process. Create automated tests for backend code. Main verification and reviewer of system/unit tests. \\ \hline
  Matthew Bradbury             & Test developer, Manual tester, Code Verifier       & Create automated tests for backend code. Manually test human detection algorithm functionality. Ensure source code follows project coding standard. Verification reviewer for the Hazard Analysis and SRS.                                                           \\ \hline
  Kyen So         & Test developer, Manual tester, Code Verifier                & Create automated tests for backend code. Manually test human outline manager functionality. Ensure source code follows project coding standards. Main verification reviewer for the Verification and Validation document.                                                                                               \\ \hline
  Tarnveer Takhtar            & Test developer, Manual tester        & Create automated tests for backend code. Manually test Kinect manager and ensure proper functionality. Verification Reviewer of Hazard Analysis and SRS                                       \\ \hline
  Dr. Gary Bone & Supervisor, SRS validator, Final reviewer &  Make sure SRS meets requirements of the project, Validate code functionality. Because Dr. Bone is the supervisor of this project, he can verify that the project is functioning as expected.\\ \hline
  \end{tabular}
  \end{table}

\subsection{SRS Verification Plan}

\wss{List any approaches you intend to use for SRS verification.  This may
  include ad hoc feedback from reviewers, like your classmates (like your
  primary reviewer), or you may plan for something more rigorous/systematic.}

\wss{If you have a supervisor for the project, you shouldn't just say they will
read over the SRS.  You should explain your structured approach to the review.
Will you have a meeting?  What will you present?  What questions will you ask?
Will you give them instructions for a task-based inspection?  Will you use your
issue tracker?}

\wss{Maybe create an SRS checklist?}

The current plan to verify our SRS involves incorporating both self-review and peer-review feedback, used in tandem with notes from our TA 
and a final read-over with our supervisor. Our team will first do a quick read-through of each other's sections and provide feedback for changes
in the form of comments on the issue. We will then incorporate the feedback we receive from our peers in another group, delivered to us via 
separate Github issues. These issues will be assigned to a single member of the team, who will have the responsibility of finishing and closing it.
Additionally, we will create issues related to the feedback we received from our TA and work on adding the corresponding changes to our SRS. 
Finally, after incorporating all the feedback received, we will host a meeting with Dr. Bone. In this meeting, the team will walk Dr. Bone through
our SRS and get his opinion on any final changes that need to be made to our requirements. Issues will be created to address the requested changes.


\subsection{Design Verification Plan}

\wss{Plans for design verification}

\wss{The review will include reviews by your classmates}

\wss{Create a checklists?}

Like the SRS, the plan for design verification includes a team review, a peer review, and a TA review. Like the SRS, the 
team review will involve team members reviewing another team member's section and commenting on changes that should be made. 
The peer review will similarly consist of another team adding Github issues to our repository and getting assigned to a team member. 
At that time, the specified team member will complete and close the issue. Additionally, we will incorporate issues raised by our TA. 

\subsection{Verification and Validation Plan Verification Plan}

\wss{The verification and validation plan is an artifact that should also be
verified.  Techniques for this include review and mutation testing.}

\wss{The review will include reviews by your classmates}

\wss{Create a checklists?}

Similarly to the previous plans, this one will consist of a team review, a peer review, and a TA review, as well as a review from our
supervisor. Just as the previous plans, the team, peer, and TA review will be conducted in the same way as previously mentioned. Additionally,
for this verification plan, we will also be including our supervisor, Dr. Bone. The team will schedule a meeting with him and go through
specific parts of the document. These parts will be our plan for what automated tests we are implementing, and Dr. Bone will have the final word
on any improvements or changes we should make before proceeding.

\subsection{Implementation Verification Plan}

\wss{You should at least point to the tests listed in this document and the unit
  testing plan.}

\wss{In this section you would also give any details of any plans for static
  verification of the implementation.  Potential techniques include code
  walkthroughs, code inspection, static analyzers, etc.}

\wss{The final class presentation in CAS 741 could be used as a code
walkthrough.  There is also a possibility of using the final presentation (in
CAS741) for a partial usability survey.}

\subsection{Automated Testing and Verification Tools}

\wss{What tools are you using for automated testing.  Likely a unit testing
  framework and maybe a profiling tool, like ValGrind.  Other possible tools
  include a static analyzer, make, continuous integration tools, test coverage
  tools, etc.  Explain your plans for summarizing code coverage metrics.
  Linters are another important class of tools.  For the programming language
  you select, you should look at the available linters.  There may also be tools
  that verify that coding standards have been respected, like flake9 for
  Python.}

\wss{If you have already done this in the development plan, you can point to
that document.}

\wss{The details of this section will likely evolve as you get closer to the
  implementation.}

For the automated testing, cppunit will be used as it provides a simple and portable way to unit test the system. When it comes to doing coverage testing it would be best to use GCov because it is compatible with VScode and easier to set up compared to other applications. The main coverage focus for the project would be MC/DC coverage because it is a good coverage test for complicated decisions which the PCD system will have to make.
Linters are also a good tool to help ensure all the code meets a certain standard that is respected within the field. For this project, Clang-Tidy will be used because it is compatible with VS code and meets the standards within the industry for C++.


\subsection{Software Validation Plan}

\wss{If there is any external data that can be used for validation, you should
  point to it here.  If there are no plans for validation, you should state that
  here.}

\wss{You might want to use review sessions with the stakeholder to check that
the requirements document captures the right requirements.  Maybe task based
inspection?}

\wss{For those capstone teams with an external supervisor, the Rev 0 demo should 
be used as an opportunity to validate the requirements.  You should plan on 
demonstrating your project to your supervisor shortly after the scheduled Rev 0 demo.  
The feedback from your supervisor will be very useful for improving your project.}

\wss{For teams without an external supervisor, user testing can serve the same purpose 
as a Rev 0 demo for the supervisor.}

\wss{This section might reference back to the SRS verification section.}

For software validation, we will be providing Dr. Bone with weekly written updates on the progress of the project. These updates
will serve as a brief validation that our software matches the aforementioned requirements outlined in the SRS. These smaller written checkups serve
as an iterative way to validate the software against the requirements by constantly getting input from Dr. Bone about new code.
Larger releases, such as code milestones or demos, will be accompanied by a meeting with Dr. Bone instead of a written update. 
These meetings will be lead by the main developers of the corresponding section and serve as the main form of software validation. 
The team will also consider peer feedback and TA/prof feedback from the demo to determine if the software accomplishes its goals.

\section{System Tests}

\wss{There should be text between all headings, even if it is just a roadmap of
the contents of the subsections.}

\subsection{Tests for Functional Requirements}

\wss{Subsets of the tests may be in related, so this section is divided into
  different areas.  If there are no identifiable subsets for the tests, this
  level of document structure can be removed.}

\wss{Include a blurb here to explain why the subsections below
  cover the requirements.  References to the SRS would be good here.}

\subsubsection{Area of Testing1}

\wss{It would be nice to have a blurb here to explain why the subsections below
  cover the requirements.  References to the SRS would be good here.  If a section
  covers tests for input constraints, you should reference the data constraints
  table in the SRS.}
		
\paragraph{Title for Test}

\begin{enumerate}

\item{test-id1\\}

Control: Manual versus Automatic
					
Initial State: 
					
Input: 
					
Output: \wss{The expected result for the given inputs.  Output is not how you
are going to return the results of the test.  The output is the expected
result.}

Test Case Derivation: \wss{Justify the expected value given in the Output field}
					
How test will be performed: 
					
\item{test-id2\\}

Control: Manual versus Automatic
					
Initial State: 
					
Input: 
					
Output: \wss{The expected result for the given inputs}

Test Case Derivation: \wss{Justify the expected value given in the Output field}

How test will be performed: 

\end{enumerate}

\subsubsection{Area of Testing2}

...

\subsection{Tests for Nonfunctional Requirements}

\wss{The nonfunctional requirements for accuracy will likely just reference the
  appropriate functional tests from above.  The test cases should mention
  reporting the relative error for these tests.  Not all projects will
  necessarily have nonfunctional requirements related to accuracy.}

\wss{For some nonfunctional tests, you won't be setting a target threshold for
passing the test, but rather describing the experiment you will do to measure
the quality for different inputs.  For instance, you could measure speed versus
the problem size.  The output of the test isn't pass/fail, but rather a summary
table or graph.}

\wss{Tests related to usability could include conducting a usability test and
  survey.  The survey will be in the Appendix.}

\wss{Static tests, review, inspections, and walkthroughs, will not follow the
format for the tests given below.}

\wss{If you introduce static tests in your plan, you need to provide details.
How will they be done?  In cases like code (or document) walkthroughs, who will
be involved? Be specific.}

\subsubsection{Area of Testing1}
		
\paragraph{Title for Test}

\begin{enumerate}

\item{test-id1\\}

Type: Functional, Dynamic, Manual, Static etc.
					
Initial State: 
					
Input/Condition: 
					
Output/Result: 
					
How test will be performed: 
					
\item{test-id2\\}

Type: Functional, Dynamic, Manual, Static etc.
					
Initial State: 
					
Input: 
					
Output: 
					
How test will be performed: 

\end{enumerate}

\subsubsection{Area of Testing2}

...

\subsection{Traceability Between Test Cases and Requirements}

\wss{Provide a table that shows which test cases are supporting which
  requirements.}

\section{Unit Test Description}

\wss{This section should not be filled in until after the MIS (detailed design
  document) has been completed.}

\wss{Reference your MIS (detailed design document) and explain your overall
philosophy for test case selection.}  

\wss{To save space and time, it may be an option to provide less detail in this section.  
For the unit tests you can potentially layout your testing strategy here.  That is, you 
can explain how tests will be selected for each module.  For instance, your test building 
approach could be test cases for each access program, including one test for normal behaviour 
and as many tests as needed for edge cases.  Rather than create the details of the input 
and output here, you could point to the unit testing code.  For this to work, you code 
needs to be well-documented, with meaningful names for all of the tests.}

\subsection{Unit Testing Scope}

\wss{What modules are outside of the scope.  If there are modules that are
  developed by someone else, then you would say here if you aren't planning on
  verifying them.  There may also be modules that are part of your software, but
  have a lower priority for verification than others.  If this is the case,
  explain your rationale for the ranking of module importance.}

\subsection{Tests for Functional Requirements}

\wss{Most of the verification will be through automated unit testing.  If
  appropriate specific modules can be verified by a non-testing based
  technique.  That can also be documented in this section.}

\subsubsection{Module 1}

\wss{Include a blurb here to explain why the subsections below cover the module.
  References to the MIS would be good.  You will want tests from a black box
  perspective and from a white box perspective.  Explain to the reader how the
  tests were selected.}

\begin{enumerate}

\item{test-id1\\}

Type: \wss{Functional, Dynamic, Manual, Automatic, Static etc. Most will
  be automatic}
					
Initial State: 
					
Input: 
					
Output: \wss{The expected result for the given inputs}

Test Case Derivation: \wss{Justify the expected value given in the Output field}

How test will be performed: 
					
\item{test-id2\\}

Type: \wss{Functional, Dynamic, Manual, Automatic, Static etc. Most will
  be automatic}
					
Initial State: 
					
Input: 
					
Output: \wss{The expected result for the given inputs}

Test Case Derivation: \wss{Justify the expected value given in the Output field}

How test will be performed: 

\item{...\\}
    
\end{enumerate}

\subsubsection{Module 2}

...

\subsection{Tests for Nonfunctional Requirements}

\wss{If there is a module that needs to be independently assessed for
  performance, those test cases can go here.  In some projects, planning for
  nonfunctional tests of units will not be that relevant.}

\wss{These tests may involve collecting performance data from previously
  mentioned functional tests.}

\subsubsection{Module ?}
		
\begin{enumerate}

\item{test-id1\\}

Type: \wss{Functional, Dynamic, Manual, Automatic, Static etc. Most will
  be automatic}
					
Initial State: 
					
Input/Condition: 
					
Output/Result: 
					
How test will be performed: 
					
\item{test-id2\\}

Type: Functional, Dynamic, Manual, Static etc.
					
Initial State: 
					
Input: 
					
Output: 
					
How test will be performed: 

\end{enumerate}

\subsubsection{Module ?}

...

\subsection{Traceability Between Test Cases and Modules}

\wss{Provide evidence that all of the modules have been considered.}
				
\bibliographystyle{plainnat}

\bibliography{../../refs/References}

\newpage

\section{Appendix}

This is where you can place additional information.

\subsection{Symbolic Parameters}

The definition of the test cases will call for SYMBOLIC\_CONSTANTS.
Their values are defined in this section for easy maintenance.

\subsection{Usability Survey Questions?}

\wss{This is a section that would be appropriate for some projects.}

\newpage{}
\section*{Appendix --- Reflection}

\wss{This section is not required for CAS 741}

The information in this section will be used to evaluate the team members on the
graduate attribute of Lifelong Learning.

\begin{enumerate}
\item Why is it important to create a development plan prior to starting the
project?\\

It is important to create a development plan prior to starting the project as
it allows most of the heavy lifting behind project planning to be done before any
work has started. It allows for expectations and workflow to be clearly defined before
any issues arise. It also creates a document that can be referenced at other times 
to avoid confusion.

\item In your opinion, what are the advantages and disadvantages of using
CI/CD?\\

Employing CI/CD allows for better issue tracking and rollbacks. Utilizing CI/CD gives the
opportunity for teams to better track individual issues and commits, leading to increased 
awareness and visibility on workflow issues. It also allows for easier time rolling back
to a previous version in case something goes wrong. 
Some disadvantages are with the conceptual depth and speed. Ensuring a specific workflow 
and constant PR reviews can slow things down as contributors have to make sure that they
are following the workflow properly and have to wait for PR reviews (when necessary). Furthermore,
it is more effort to set up, both in the codebase and conceptually. The process has to be 
talked through and understood by all team members.

\item What disagreements did your group have in this deliverable, if any,
and how did you resolve them?\\

Our group mainly debated how to set up our GitHub workflows. Initially, we considered using individual branches for each issue, but we ultimately decided on a 
more streamlined approach with two revision branches and separate forks. This allows us to effectively manage pull requests for merging feature changes into the codebase. 
We reached this agreement after discussing the benefits of clarity and collaboration in our development process.

\item What went well while writing this deliverable? \\

This deliverable allowed for the group to be able to discuss standard goals vs stretched goals. Everyone contributed their ideas and we were able to come to a clear conclusion 
when it came to the goals and problem statement of the project. It also allowed us all to see where the project is headed and how we should properly prepare
 ourselves so that we can achieve our goals. 

\item What pain points did you experience during this deliverable, and how
did you resolve them?\\

The biggest pain point during this deliverable was being able to decide which goals were too ambitious and outside our design scope. Some goals such as the aspect of outlining the human 
in the environment were broken up. There was a deep discussion on how to properly decide the goals, but at the end the  group got a better understanding of the project.

\item How did you and your team adjust the scope of your goals to ensure
they are suitable for a Capstone project (not overly ambitious but also of
appropriate complexity for a senior design project)?\\

Because our project is presented by a professor, the team already had a pretty clear understanding of the goals that needed to be achieved for our project to be considered a success.
 The only issue was trying to ensure that the goals can be properly broken down so that a goal that seemed a bit ambitious could be broken into something that seems doable. For example, 
 the human detection is broken into the Minimal and viable product goal and then also extended into the stretch goal. This was an important discussion that the group had to ensure that we 
 deemed our goals to be doable.

\item What knowledge and skills will the team collectively need to acquire to successfully complete this capstone project?  Examples of possible knowledge to acquire include domain specific knowledge 
from the domain of your application, or software engineering knowledge, mechatronics knowledge or computer science knowledge.  Skills may be related to technology, or writing, or presentation, 
or team management, etc.  You should look to identify at least one item for each team member.

Every member of the group needs to acquire knowledge on Computer Vision and understanding how pcd files operate. Overall these are big topics and so the basics should be acquired by every member, but 
some specifics would be broken down to different parts for each member. Tarnveer and Matthew would be assigned to understanding how to track people on screen and map them on the screen. Harman and Kyen 
would be working on reading in the pcd files and understanding the PCL. The PCL will explore on aspects of boxing the points on screen. Tarnveer would also be responsible for understanding
the real time coding aspect in c++. Matthew would also be assigned to acquire knowledge on improving the human outline based on better data. Harman will be assigned to understanding how to cancel out noise
from the data set. Kyen will have to understand how to find the person based off the pcd and understand how to properly read the files.

\item For each of the knowledge areas and skills identified in the previous question, what are at least two approaches to acquiring the knowledge or mastering the skill?  Of the identified approaches, which
will each team member pursue, and why did they make this choice?

One approach for acquiring the knowledge would be to use the provided documentation for the libraries (PCL and OpenCV). This would provide a good fundamental understanding for the important aspects of the project. 

Another approach would be to just watch videos on the specific topics and try to understand from there. This would be able to provide a more visual explanations for the topics.

Tarnveer: use documentation and videos
Matthew: use documentation and videos 
Kyen: use documentation
Harman: use documentation and videos

\item What went well while writing this deliverable? 

The deliverable was straightforward. We had a rough idea of the main hazards within our project and 
tried to make sure that we covered the main scope. The document writing was split between all of us.
The document is pretty straightforward and we as a group were able to talk over the different sections
and divide up the work.

\item What pain points did you experience during this deliverable, and how
did you resolve them?

The biggest pain point was probably discussing what would be some assumptions we had to make, but we were able to come to an agreement by communicating our points of why or why not.

\item Which of your listed risks had your team thought of before this
deliverable, and which did you think of while doing this deliverable? For
the latter ones (ones you thought of while doing the Hazard Analysis), how
did they come about?

All the risks were mainly thought of before the deliverable. We knew that we needed to ensure that the 
offline file is the correct format and that the system needs to make sure it is working with a Kinect sensor and not something else.

The privacy risk was something we thought of at the informal interview.

\item Other than the risk of physical harm (some projects may not have any
appreciable risks of this form), list at least 2 other types of risk in
software products. Why are they important to consider?

Could be some performance risks and making sure that the performance of the software meets the goals/requirements for the project.
Another risk could be in terms of privacy. The application is capturing sensitive data and so its important on how the application handles this data.

\item What went well while writing this deliverable?

This deliverable was relatively painless and straightforward. We had already started thinking about testing plans
during our SRS deliverable, specifically section S.6. In this section we detailed a brief VnV plan, including system testing
and unit testing. This set up the basic outline for this deliverable, it being an extension of what we already wrote/thought about.

\item What pain points did you experience during this deliverable, and how
  did you resolve them?

One pain point we had during this deliverable was editing requirements from the SRS. When creating the traceability matrix, we noticed
that some of the requirements from the SRS document were overlapping or in the wrong spot. We held a meeting as a group to sort this out 
and reach a consensus on which requirements should stay, should be changed, or should be deleted.

\item What knowledge and skills will the team collectively need to acquire to
successfully complete the verification and validation of your project?
Examples of possible knowledge and skills include dynamic testing knowledge,
static testing knowledge, specific tool usage, Valgrind etc.  You should look to
identify at least one item for each team member.

In order to properly complete the verification and validation of our project, some skills will need to Be
acquired. Firstly, we will have to familiarize ourselves with cppunit, as majority of our testing knowledge from
previous courses is in Java or Python. Additionally, because of this, we will have to learn how to use GCov in order to
accurately figure out our code coverage. On the topic of code coverage, we will also need to brush up on our coverage 
definitions that were learnt in our testing course. Finally, we will need to implement linters to check our code on github
before merge.

\item For each of the knowledge areas and skills identified in the previous
question, what are at least two approaches to acquiring the knowledge or
mastering the skill?  Of the identified approaches, which will each team
member pursue, and why did they make this choice?

For these skills, there are a few ways to approach acquiring the knowledge. With new skills, utilizing Youtube and online tutorials
are a good way to quickly learn the basics and proper implementation of new techniques. For older skills, i.e. ones that we have learnt
previously but haven't used in a while, we can go back to old projects/lectures and relearn the information.

Matthew will find online tutorials to learn about cppunit. This is because he has no experience with creating automated testing in c++, and needs 
to start off by learning the basics.

Tarnveer will find online tutorials to learn about cppunit and c++ linters on Github. This is for the same reason as above; he has no experience
implementing testing in c++ or adding linters to Github for PRs.

Harman will go back to our old 3SO3 notes in order to relearn MC/DC coverage. This is because he had implemented MC/DC coverage and checks 
in that course, but in Java. Since he has implemented this before, he is familiar with the content and should be relatively painless to 
relearn the content.

Kyen will find online tutorials for learning about GCov. For similar reasons as Matthew and Tarnveer, this is because he has no prior experience with 
this specific coverage tool. 

\item What went well while writing this deliverable? 

  Everyone on the team was on track with their sections of the assignment and we were able to thick of better ways to break up some modules to make more sense.

\item What pain points did you experience during this deliverable, and how did you resolve them?

  Getting used to the new year and so it was a slow start trying to get back into the flow, but once we started working it came back.

\item Which of your design decisions stemmed from speaking to your client(s)or a proxy (e.g. your peers, stakeholders, potential users)? For those thatwere not, why, and where did they come from?

  Most of the module break up comes from talking to our client becuase they helped us focus on their vision for the project but making the inputs and specific variables were all done independently.

\item While creating the design doc, what parts of your other documents (e.g.
  requirements, hazard analysis, etc), it any, needed to be changed, and why?

  For now no real document had to be changed becuase the structure for this assignment was thought of before through the many client meets. This allowed for a strong structure.

\item What are the limitations of your solution?  Put another way, given
  unlimited resources, what could you do to make the project better? (LO\_ProbSolutions)

  With unlimited resources the ability to capture better imaging with the kinect would allow for a faster and more precise human detection algorithm. Maybe also being able to better maximize the human detection to better fit a humaniod shape.
\item Give a brief overview of other design solutions you considered.  What
  are the benefits and tradeoffs of those other designs compared with the chosen
  design?  From all the potential options, why did you select the documented design?
  (LO\_Explores)

  Other design implications would just involve taking a different approach to creating the algorithm. The issue with for example a solution that does not use hue or skin color is limiting our ability to full captalize on the fact that the sensor picks up RGB as well.

\end{enumerate}

\begin{enumerate}
  \item What went well while writing this deliverable? 
  \item What pain points did you experience during this deliverable, and how
    did you resolve them?
  \item What knowledge and skills will the team collectively need to acquire to
  successfully complete the verification and validation of your project?
  Examples of possible knowledge and skills include dynamic testing knowledge,
  static testing knowledge, specific tool usage, Valgrind etc.  You should look to
  identify at least one item for each team member.
  \item For each of the knowledge areas and skills identified in the previous
  question, what are at least two approaches to acquiring the knowledge or
  mastering the skill?  Of the identified approaches, which will each team
  member pursue, and why did they make this choice?
\end{enumerate}

\end{document}