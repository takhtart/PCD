\documentclass{article}

\usepackage{tabularx}
\usepackage{booktabs}

\title{Reflection and Traceability Report on \progname}

\author{\authname}

\date{}

\input{../Comments}
%% Common Parts

\newcommand{\progname}{PCD: Partially Covered Detection of Obscured People using Point Cloud Data} % PUT YOUR PROGRAM NAME HERE
\newcommand{\authname}{Team \#14, PCD
\\ Tarnveer Takhtar
\\ Matthew Bradbury
\\ Harman Bassi
\\ Kyen So} % AUTHOR NAMES                  

\usepackage{hyperref}
    \hypersetup{colorlinks=true, linkcolor=blue, citecolor=blue, filecolor=blue,
                urlcolor=blue, unicode=false}
    \urlstyle{same}
                                


\begin{document}

\maketitle

\plt{Reflection is an important component of getting the full benefits from a
learning experience.  Besides the intrinsic benefits of reflection, this
document will be used to help the TAs grade how well your team responded to
feedback.  Therefore, traceability between Revision 0 and Revision 1 is and
important part of the reflection exercise.  In addition, several CEAB (Canadian
Engineering Accreditation Board) Learning Outcomes (LOs) will be assessed based
on your reflections.}

\section{Changes in Response to Feedback}

The issues linked below contain all the reqested information.


\subsection{Problem Statement and Development Plan}

Problem Statement:\\

\href{https://github.com/takhtart/PCD/issues/65}{65}\\
\href{https://github.com/takhtart/PCD/issues/66}{66}\\
\href{https://github.com/takhtart/PCD/issues/67}{67}\\
\href{https://github.com/takhtart/PCD/issues/68}{68}\\
\href{https://github.com/takhtart/PCD/issues/70}{70}\\
\href{https://github.com/takhtart/PCD/issues/71}{71}\\
\href{https://github.com/takhtart/PCD/issues/72}{72}\\
\href{https://github.com/takhtart/PCD/issues/73}{73}\\

Development Plan:\\

\href{https://github.com/takhtart/PCD/issues/74}{74}\\
\href{https://github.com/takhtart/PCD/issues/75}{75}\\
\href{https://github.com/takhtart/PCD/issues/76}{76}\\
\href{https://github.com/takhtart/PCD/issues/77}{77}\\
\href{https://github.com/takhtart/PCD/issues/78}{78}\\
\href{https://github.com/takhtart/PCD/issues/79}{79}\\

\subsection{SRS and Hazard Analysis}

SRS Changes:\\
\href{https://github.com/takhtart/PCD/issues/80}{80}\\
\href{https://github.com/takhtart/PCD/issues/81}{81}\\
\href{https://github.com/takhtart/PCD/issues/82}{82}\\
\href{https://github.com/takhtart/PCD/issues/83}{83}\\
\href{https://github.com/takhtart/PCD/issues/84}{84}\\
\href{https://github.com/takhtart/PCD/issues/85}{85}\\
\href{https://github.com/takhtart/PCD/issues/86}{86}\\
\href{https://github.com/takhtart/PCD/issues/87}{87}\\
\href{https://github.com/takhtart/PCD/issues/88}{88}\\
\href{https://github.com/takhtart/PCD/issues/89}{89}\\
\href{https://github.com/takhtart/PCD/issues/90}{90}\\
\href{https://github.com/takhtart/PCD/issues/91}{91}\\
\href{https://github.com/takhtart/PCD/issues/112}{112}\\

Hazard Analysis:\\

\href{https://github.com/takhtart/PCD/issues/92}{92}\\
\href{https://github.com/takhtart/PCD/issues/93}{93}\\
\href{https://github.com/takhtart/PCD/issues/94}{94}\\
\href{https://github.com/takhtart/PCD/issues/95}{95}\\
\href{https://github.com/takhtart/PCD/issues/96}{96}\\


\subsection{Design and Design Documentation}

MG:\\
\href{https://github.com/takhtart/PCD/issues/104}{104}\\
\href{https://github.com/takhtart/PCD/issues/105}{105}\\

MIS:\\
\href{https://github.com/takhtart/PCD/issues/106}{106}\\
\href{https://github.com/takhtart/PCD/issues/107}{107}\\

\subsection{VnV Plan and Report}

VnV Plan:\\
\href{https://github.com/takhtart/PCD/issues/97}{97}\\
\href{https://github.com/takhtart/PCD/issues/98}{98}\\
\href{https://github.com/takhtart/PCD/issues/99}{99}\\
\href{https://github.com/takhtart/PCD/issues/100}{100}\\
\href{https://github.com/takhtart/PCD/issues/101}{101}\\
\href{https://github.com/takhtart/PCD/issues/102}{102}\\
\href{https://github.com/takhtart/PCD/issues/103}{103}\\
\href{https://github.com/takhtart/PCD/issues/113}{113}\\

VnV Report:\\

\href{https://github.com/takhtart/PCD/issues/118}{118}\\
\href{https://github.com/takhtart/PCD/issues/119}{119}\\
\href{https://github.com/takhtart/PCD/issues/122}{122}\\
\href{https://github.com/takhtart/PCD/issues/123}{123}\\
\href{https://github.com/takhtart/PCD/issues/124}{124}\\

\section{Challenge Level and Extras}

\subsection{Challenge Level}

The challenge level of this project is currently set to Advanced/Research
level. The project consists of topics that aren’t covered within previous year
courses, and the majority of the project doesn’t fit into the group’s current do-
main knowledge on computer vision and point cloud data manipulation. There
is a learning curve with the computer vision aspect of the program, as we will
have to quickly learn and apply libraries and techniques that are novel to us.


\subsection{Extras}

As our project is advanced, the only extra will be a research manual.

\section{Design Iteration (LO11 (PrototypeIterate))}

The project is exploritory by nature and thus, the evolution of the project reflects the new strategies that we learned throughout the process. 
The strategies that we employed early on were changed as we discovered what worked and what didn't, as well as received advice from our supervisor/TA.
The POC started with using the largest cluster in the point cloud to detect humans. We then decided to move away from this and use skin detection as advised by our supervisor.
This would help with issues that we found with our intial method. This led to us exploring region growing methods after successfully detecting skin points on a person. 
On top of this, we repurposed noise removal techniques used in our POC to further reduce false positives that we noticed were happening during skin detection.
Following this, we noticed issues with our skin detection falsely pinpointing skin on the background, which led us to developing a plane removal process. Finally, 
our supervisor recommended a broader approach to human location prediction: instead of trying to segment the person we should focus on a wider-ranged approach and perform a looser estimation based on the relative size of a person.

\section{Design Decisions (LO12)}

Limitations: Due to the Kinect having a limited range of vision, we were unable to fully capture all data points in an open environment. This,
combined with the fact that the infrared light for depth sensing is easily disrupted by environmental factors, meant that we were limited to designing 
a process that worked in a very controlled and closed environment.

Constraints: For this project, our proposed constraints were to not use machine learning and to only use the Kinect sensor.
Because we weren't able to use machine learning, we couldn't explore any approach utilizing a trained model for detection. Since
we could only use a Kinect sensor, we had to make design decisions around using the Kinect library and the specific format that the Kinect would output. 

Assumptions: The assumptions that we made for this project were largely to ensure that we could work within the contraints and limitations mentioned above. 
This included limiting the technology we could use as well as the useable environment of our process.



\section{Economic Considerations (LO23)}

There is a definite market for our product. It has some very widely applicable use cases in robotics. However, the current major drawback is the 
stage of development that it is at. In order to get this product ready for market, it would require a sizeable software team, of around 4, working for 
about 2 years. With the average salary of a software engineer being 97,000 \$CAD per year, it would cost about 776,000 \$CAD to achieve a marketable product.
In order to attract users, we would have to highlight the performance and memory cost of our solution, compared to a similar product that uses machine learning. 
Our potential userbase includes professors or graduate students researching in the robotics space, as well as companies that develop in the robotics space.

\section{Reflection on Project Management (LO24)}

\subsection{How Does Your Project Management Compare to Your Development Plan}

With regards to our development plan, the team meetings, team communication, team member roles, and workflow plan
were largely followed throughout the term. The main things that changes were our communication with our supervisor and our
planned technology. With our supervisor, we noticed that we were communicating less than needed and attempted to have more communication as the year went on. 
In terms of technologies, we moved away from our initial guess at what software we would use as we gained more insight into the problem we were attempting to solve.

\subsection{What Went Well?}

Everyone completed their tasks on time, and whenever assistance was needed they reached out to ensure that everything would be covered. As for technology,
groupmates were each assigned specific topics to learn about throughout the course of the project and each teammate fufilled this duty. Each team member put in the
effort needed to understand the different technologies for the project, providing an end product that the team is satisified with.

\subsection{What Went Wrong?}

In terms of technology, we used none of the technology that we guessed we would be using. This was
not a huge deal as our project is largely exploratory, but it lead to many corrections needing to be made in documentation.
We also had to add a new requirement from our supervisor later on in the course, which caused more work that needed to be done in a shorter amount of time. 

\subsection{What Would you Do Differently Next Time?}

For our next project, we would ensure that all of our requirements are clearly defined from the beginning. Much of the time crunch and workload
later on in the project was due to the aforementioned added requirement that was "sprung" on us. In order to avoid this, we would need to clearly extract all the 
requirements from our client before starting and not tolerate drastic changes to the plan, or try to compromise with what we are adding.

\section{Reflection on Capstone}


\subsection{Which Courses Were Relevant}

The course that helped with building the goals and requirements has been the software requirements course. 
An important step when it comes to working on big projects is really understanding the project to the core 
before anything is done and thats what this course helped with. Another course that helped is our testing
course which will really help when testing the software (unit tests and functionality tests). Finally, the 
multiple coding classes that were taking for example OOP that provided knowledege on how to write clean code and 
how to implement multiple alogorithms.

\subsection{Knowledge/Skills Outside of Courses}

We definitely needed to acquire skills and knowledge that lay outside information acquired in class. Our project was 
centred around computer vision, of which no member of our group had prior experience in. We learnt much about the topic of computer
vision as a whole, as well as associated libraries, such as PCL and OpenCV. We also had to work with using a Kinect and point cloud data, both 
of which were completely foreign to the team. 

\end{document}